\section{Introduction}

Many problems in biophysics involve the interaction of an incompressible fluid and an
immersed elastic interface. The solution to these problems can be approximated using the
immersed boundary (IB) method, which was developed by Charles Peskin to simulate blood
flow through the heart \cite{Peskin:1972wa}. The relative ease by which the IB method can
be incorporated into a Navier-Stokes solver has led to its popularity in myriad
applications (see, e.g. \cite{Iaccarino:2005ii,Griffith:2020hi} and references therein).
The IB method couples the equations governing fluid velocity and pressure to those
governing the interface movement and elastic forces via operations we will refer to as
\term{interpolation} and \term{spreading}. Fluid velocities are interpolated to points on
the interface, and forces on the interface are spread to the fluid. Fluid properties are
discretized on an Eulerian grid with fixed Cartesian locations. The interface is
represented by a set of Lagrangian points whose Cartesian locations vary in time. This
presents a problem for effective parallelization of interpolation and spreading, as these
operators must be reconstructed at each timestep to account for the motion of the
interface. The Eulerian and Lagrangian points, on the other hand, can be treated as fixed
for several timesteps, if not for the entirety of a simulation. We may therefore treat
solving the fluid equations and computing elastic forces as an implementation detail and
focus primarily on the interpolation and spreading operations.

We are interested primarily in the case of neutrally buoyant, elastic immersed structures
that move at the local fluid velocity, but do not impose it. In particular, we aim to
simulate whole blood, which is composed of red blood cells (RBCs) and platelets immersed
in blood plasma, within a vessel lined by endothelial cells. The cells are elastic, and
as they deform with the flow of the enveloping fluid, impose a force on the fluid.
Approximately 40\% of the volume in healthy human blood is occupied by RBCs, and so even
small domains may require tens or hundreds of thousands of points to discretize these
cells for use within the IB method.

To take advantage of modern computing architectures, with ever-increasing numbers of
processors, it is necessary to develop parallel algorithms for the IB method.
McQueen and Peskin \cite{McQueen:1997kw} present a domain decomposition scheme to
parallelize the interpolation and spreading operations on the Cray C-90 computer with
shared memory and a modest number of vector processors. Their results illustrate the need
for a fast interpolation and spreading: even parallelized, they spend roughly half of the
wall clock time spreading and interpolating. Fai, \textit{et al.} \cite{Fai:2013do}
adapted this domain decomposition scheme for use on a general purpose graphical
processing unit (GPGPU; GPU for short). Patel \cite{Patel:2012tc} parallelized spreading
on a GPU by processing one Lagrangian point at a time. Because neither of these
approaches can concurrently process any arbitrarily Lagrangian points, it is easy to find
cases for which they perform poorly. An alternative is to distribute work among several
devices, each with their own memory. This idea underpins the popular IBAMR library
\cite{Griffith:2007uk,Griffith:2007do,Griffith:2009gg, Griffith:2011gi, Griffith:2017id},
which also adaptively refines the mesh around the immersed structure. With proper
load-balancing, this allows a serial algorithm to process a smaller portion of work.
A cluster of multicore devices, however, will not be used effectively without a
shared-memory parallel algorithm. The cuIBM \cite{Layton:2011um} and PetIBM
\cite{Chuang:2018ej} libraries implement an adaptive IB method for prescribed motion on
single- and multi-GPU architectures, respectively, by explicitly constructing the
spreading and interpolation operators. These operators are sparse, but the sparsity
pattern of the resulting spreading matrix does not always lend itself well to
parallelization.

GPUs have severe restrictions on their parallelization. They use single instruction,
multiple data (SIMD) parallelism, in which each computational unit, or thread, executes
the same instruction on its own data. Threads attempting to perform different operations
do so serially. On the other hand, multicore CPUs can perform different tasks in parallel
without issue. Some CPUs even have support for certain SIMD operations or instruction
pipelining. Concurrency on the GPU can be limited by exceeding the amount of shared
memory, which is shared among a group of threads, or register memory, which is accessible
only to a single thread. The alternative is to use global memory, which is slow in
general, but faster when accesses are sequential (or ``coalesced''). The restrictions on
the GPU lead us to treat the GPU as the lowest common denominator; an effective algorithm
for the GPU should translate well to other shared-memory architectures.

Here, we discuss the parallelization of interpolation and introduce a new parallelization
for the spreading operation. We demonstrate that the concurrency of these algorithms
scales with the number of IB points, independently of the Eulerian grid. The algorithms
eschew the explicit construction of spreading and interpolation operators, instead
dividing the operations into trivially parallelizable tasks and parallel primitives. They
are therefore suitable for use on GPGPUs and by extension, CPUs. We begin with an
introduction to the IB method, and the role of the interpolation and spreading
operations.

% vim: cc=90 tw=89
