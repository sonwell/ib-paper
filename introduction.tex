\section{Introduction}

Many problems in biophysics involve the interaction of an incompressible fluid and an
immersed elastic interface. The solution to these problems can be approximated using the
immersed boundary (IB) method, which was developed in~\cite{Peskin:1972wa} to simulate
blood flow through the heart. The relative ease by which the IB method can be
incorporated into a Navier-Stokes solver has led to its popularity in myriad applications
(see, e.g.~\cite{Iaccarino:2005ii,Griffith:2020hi} and references therein).
The IB method couples the equations governing fluid velocity and pressure to those
governing the interface movement and elastic forces via operations we will refer to as
\term{interpolation} and \term{spreading}. Fluid velocities are interpolated to points on
the interface, and forces on the interface are spread to the fluid. Fluid properties are
discretized on a fixed Eulerian grid. The interface is represented by a set of mobile
Lagrangian points. This presents a problem for effective parallelization of interpolation
and spreading, as these operators must be reconstructed at each timestep to account for
the motion of the Lagrangian points relative to the Eulerian grid. The Eulerian and
Lagrangian points, on the other hand, can be treated as fixed in their respective
coordinate spaces for several timesteps, if not for the entirety of a simulation. We may
therefore treat solving the fluid equations and computing elastic forces as an
implementation detail and focus primarily on the interpolation and spreading operations.

We are interested in the case of neutrally buoyant, elastic immersed structures that move
at the local fluid velocity, but do not impose it. In particular, we aim to simulate
whole blood, which is composed of red blood cells (RBCs) and platelets immersed in blood
plasma, within a vessel lined by endothelial cells. The cells are elastic, and as they
deform with the flow of the enveloping fluid, impose a force on the fluid. Approximately
40\% of the volume in healthy human blood is occupied by RBCs, and so even small domains
may require tens or hundreds of thousands of points to discretize these cells for use
within the IB method.

To take advantage of modern computing architectures, with ever-increasing numbers of
processors, it is necessary to develop parallel algorithms for the IB method.~%
\cite{McQueen:1997kw} present a domain decomposition scheme to parallelize the
interpolation and spreading operations on the Cray C-90 computer with shared memory and a
modest number of vector processors. Their results illustrate the need for a fast
interpolation and spreading: even parallelized, they spend roughly half of the wall clock
time spreading and interpolating.~\cite{Fai:2013do} adapted this domain decomposition
scheme for use on a general purpose graphical processing unit (GPGPU; GPU for short).~%
\cite{Patel:2012tc} parallelized spreading on a GPU by processing one Lagrangian point at
a time. Because neither of these approaches can concurrently process arbitrary Lagrangian
points, it is easy to find cases for which they perform poorly. An alternative is to
distribute work among several devices, each with its own memory. This idea underpins the
popular IBAMR library (see~\cite{Griffith:2007uk,Griffith:2007do,Griffith:2009gg,
Griffith:2011gi,Griffith:2017id}), which also adaptively refines the mesh around the
immersed structure. With proper load-balancing, this allows a serial algorithm to process
a smaller portion of work. A cluster of multicore devices, however, will not be used
effectively without a shared-memory parallel algorithm. The cuIBM (\cite{Layton:2011um})
and PetIBM (\cite{Mesnard:2017te,Chuang:2018ej}) libraries implement an adaptive IB
method for prescribed motion on single- and multi-GPU architectures, respectively. The
authors demonstrate their method on a few two-dimensional test problems. Their
implementation explicitly construct the spreading and interpolation operators,
which are sparse, but the sparsity pattern of the spreading matrix does not always lend
itself well to parallelization.

GPUs have restrictions on their parallelization. They use single instruction,
multiple data (SIMD) parallelism, in which each computational unit, or thread, executes
the same instruction on its own data. Concurrency on the GPU is typically limited by the
amount of shared memory, which is shared among a group of threads, and register memory,
which is accessible only to a single thread. The alternative is to use global memory,
which is slow in general, but faster when accesses are sequential (or ``coalesced'').
These restrictions on the GPU imply that an effective algorithm for the GPU translates
well to other shared-memory architectures, such such as multicore CPUs, which support
parallelism, SIMD instructions, advanced instruction pipelining, and out-of-order
execution. We therefore develop parallel spreading and interpolation algorithms
applicable to both GPUs and multicore CPUs. The success of our new algorithms relies in
dividing these operations into trivially parallelizable tasks and parallel primitives.

The remainder of this paper is organized into four sections. Section~\ref{sec:ib} gives
an overview of the IB method, and describes the role of the interpolation and spread
operations. In Section~\ref{sec:parallel}, we discuss the parallelization of
interpolation and introduce a new parallelization for the spreading operation. In
Section~\ref{sec:results}, we demonstrate that the concurrency of these algorithms scales
with the number of IB points, independently of the Eulerian grid. We confirm the
convergence of the IB method with these new algorithms. We illustrate the suitability of
our algorithms to both GPUs and multicore CPU architectures with both weak and strong
scaling tests. Finally, we summarize our findings in Section~\ref{sec:summary}.

% vim: cc=90 tw=89
