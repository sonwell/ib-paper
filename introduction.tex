\section{Introduction}

Many problems in biophysics involve the interaction of an incompressible fluid
and an immersed elastic interface. The solution to these problems can be
approximated using the immersed boundary (IB) method, which was developed by
Peskin to simulate blood flow through the heart [@Peskin:1972wa]. This method
couples the equations governing fluid velocity and those governing elastic
forces via operations we will refer to as \textit{interpolation} and
\textit{spreading}. We treat solving the fluid equations and computing elastic
forces as an implementation detail; we assume that they are parallelizable and
limit our focus to the interpolation and spreading operations.

McQueen and Peskin present a domain decomposition scheme to parallelize these
operations on the Cray C-90 computer with a modest number of vector processors
[@McQueen:1997kw]. Unfortunately, the trend in computing is to use \emph{more}
processors rather than \emph{faster} processors. They also illustrate the need
for a fast interpolation and spreading -- even parallelized, they spend roughly
half of the wall clock time spreading and interpolating. Here, we introduce
a novel parallelization scheme for the interpolation and spreading operations.
We compare this to the method of Peskin and McQueen and a variant of their
ideas on a massively parallel general purpose graphical processing unit
(GPGPU).
