\section{Numerical results}

Here we describe two types of test: unstructured IB points, in which points are placed
randomly in the domain and generate a force independently from the other IB points, and
structured IB points, in which the points comprise an elastic structure and forces are
generated based on the configuration of the points as a whole. For these tests, we use
a $16\um\times16\um\times16\um$ triply periodic domain with an initially shear-like flow,
$\u=(0,\,0,\,\shear(y-8\um))$, with shear rate $\shear$. Tests use a shear rate of
$1000\si{\per\second}$ unless otherwise noted. This flow has a sharp transition at the
periodic boundary $y=0\um$, so a background force is added to maintain this transition
and so that the initial flow is also the steady flow in the absence of other forces, as
in \cite{Fai:2013do}.

Serial and multicore CPU tests were performed on a single node with 48 Intel{\reg}
Xeon{\reg} CPU E5-2697 v2 2.70\si{\giga\hertz} processors and 256 GB of RAM running
CentOS Linux release 7.7.1908 (x86\_64). Parallel CPU implementations use Intel's OpenMP
library, \texttt{libiomp5}. GPU tests used the same node with an NVIDIA{\reg} Tesla{\reg}
K80 ($2\times$GK210 GPU with 13 823.5\si{\mega\hertz} multiprocessors and 12 GB of global
memory each). Only one of the GK210 GPUs was used. The CPU code was written in C++17 and
the GPU code was written in C++/CUDA and used version 9.2 CUDA libraries. Both the CPU
and GPU code were compiled using \texttt{clang} version 7.0.1. All tests are performed in
double precision. We begin with tests using unstructured IB points, for which both of
these architectures were used.

\subsection{Unstructured IB points}\label{sec:unst}

Consider a set of $n$ IB points randomly placed in the domain described above. The IB
points are imagined to be tethered to their initial positions. The fluid solver is not
invoked for these tests. Instead, at each timestep, we interpolate the fluid velocity to
each of the IB points and predict updated positions for the IB points. Using these
predicted positions, we compute a Hookean force for each IB point with spring constant
$0.01\si{dyn\per\centi\meter}$. We spread these forces from the predicted positions to
the fluid grid, but do not use them to update the fluid velocity. This ensures that the
points do not settle into a steady position so the spreading and interpolation operations
receive new data each timestep. Finally, we interpolate the velocity to the positions of
the IB points at the beginning of the timestep again and update the position of the IB
points. While the interpolated velocities are the same as those computed at the beginning
of the timestep, this is done by analogy to the fluid solver, which interpolates fluid
velocities twice and spreads forces once per timestep.

We use this test to compare the performance of the parallel algorithms to their serial
counterparts and, for the spreading variants, to each other.


\subsection{Dependence on background grid}

\subsubsection{Strong scaling}\label{sec:unst-strong}

It is commonly the case that one wishes to employ parallelization to improve runtimes for
a problem of interest. To illustrate this improvement, we now consider how runtime varies
for the test problem with $n=2^{16}$ IB points, a grid refinement of 64 (grid size
$h=0.25\um$), and Algorithm \ref{algo:otf-spread} with different numbers of threads. We
use up to 32 threads on the CPU and 64--4096 threads on the GPU. For a fixed problem, we
ideally wish to see the runtime using $2p$ threads to be half of that using $p$ threads.
In other words, using twice as many threads yield an ideal speedup of 2.


%\begin{table}[h]
%\begin{center}
%\bgroup
%\renewcommand{\arraystretch}{1.7}
%\begin{tabular}{cccc}
%                                                                                  \toprule
%    $p$  & device & \titletable{interpolate}{2000} & \titletable{spread}{1000} \\ \midrule
%         & CPU    & $1.30763 \pm 0.01795$          & $1.33840 \pm 0.01221$     \\
%    1    & CPU    & $1.30437 \pm 0.00986$          & $1.51528 \pm 0.02790$     \\
%    2    & CPU    & $0.67160 \pm 0.00605$          & $0.80398 \pm 0.02135$     \\
%    4    & CPU    & $0.35833 \pm 0.00284$          & $0.44432 \pm 0.01592$     \\
%    8    & CPU    & $0.21166 \pm 0.00773$          & $0.28799 \pm 0.04476$     \\
%    16   & CPU    & $0.10044 \pm 0.00258$          & $0.15393 \pm 0.01553$     \\
%    32   & CPU    & $0.07411 \pm 0.00068$          & $0.12394 \pm 0.01759$     \\
%    64   & GPU    & $0.80977 \pm 0.00478$          & $0.96698 \pm 0.00047$     \\
%    128  & GPU    & $0.43929 \pm 0.00352$          & $0.49359 \pm 0.00064$     \\
%    256  & GPU    & $0.22434 \pm 0.00223$          & $0.26208 \pm 0.00145$     \\
%    512  & GPU    & $0.11258 \pm 0.00150$          & $0.14286 \pm 0.00177$     \\
%    1024 & GPU    & $0.05681 \pm 0.00097$          & $0.08252 \pm 0.00150$     \\
%    2048 & GPU    & $0.03074 \pm 0.00071$          & $0.05351 \pm 0.00103$     \\
%    4096 & GPU    & $0.01689 \pm 0.00047$          & $0.03905 \pm 0.00093$     \\ \bottomrule
%\end{tabular}
%\egroup
%\caption{%
%    Strong scaling results for OTF and grid spacing $h = 0.5\um$ (a grid
%    refinement of 64) for $2^{16}$ randomly placed IB points in a
%    $16\um\times16\um\times16\um$ triply periodic domain. The first row
%    are reference values for serial Algorithms \ref{algo:par-interp} and
%    \ref{algo:serial-spread}.
%}
%\end{center}
%\end{table}

%1  & CPU & $1.29442 \pm 0.00359$ & $1.50377 \pm 0.00234$
%2  & CPU & $0.65481 \pm 0.00271$ & $0.92804 \pm 0.00147$
%4  & CPU & $0.34712 \pm 0.00166$ & $0.44697 \pm 0.00148$
%8  & CPU & $0.18004 \pm 0.00084$ & $0.24580 \pm 0.00082$
%16 & CPU & $0.09969 \pm 0.00075$ & $0.14895 \pm 0.00136$
%32 & CPU & $0.07371 \pm 0.00083$ & $0.12225 \pm 0.00143$

\begin{figure}[htbp]
\begin{tikzpicture}
\begin{groupplot}[
    group style={group name=unst-strong, group size=1 by 2},
    width=0.45\textwidth
]
\nextgroupplot[
        xmin=0.70710678118,  % approx. 1/√2
        xmax=45.2548339959,  % approx 32√2
        xmode=log,
        log basis x=2,
        ymode=log,
        ymin=0.5,
        ymax=128,
        log basis y=2,
        log origin=infty,
        width=0.45\textwidth,
        height=0.45\textwidth,
        axis lines=center,
        ylabel={speedup},
        xlabel near ticks,
        ylabel near ticks,
        xtick={1, 2, 4, 8, 16, 32},
        ytick={1, 4, 16, 64},
        legend style={at={(0.5, 0.9)}, anchor=center, draw=none,
                      /tikz/every even column/.append style={column sep=5pt}},
        legend cell align={left},
        legend columns=1
    ]
    \addplot+[only marks, mark=diamond*, color=tol/vibrant/magenta, mark options={scale=2, fill=tol/vibrant/magenta}] coordinates {%
        (1 , {1.30763/1.29442})
        (2 , {1.30763/0.65481})
        (4 , {1.30763/0.34712})
        (8 , {1.30763/0.18004})
        (16, {1.30763/0.09969})
        (32, {1.30763/0.07371})
    }; \label{plot:unst-strong-interp}
    \addplot+[only marks, mark=triangle*, color=tol/vibrant/blue, mark options={scale=2, fill=tol/vibrant/blue}] coordinates {%
        (1 , {1.33840/1.50377})
        (2 , {1.33840/0.92804})
        (4 , {1.33840/0.44697})
        (8 , {1.33840/0.24580})
        (16, {1.33840/0.14895})
        (32, {1.33840/0.12225})
    }; \label{plot:unst-strong-spread}
    \addplot+[no marks, dashed, black, domain=1:32] {x^0.93};
    \addplot+[no marks, dotted, black, domain=1:32] {0.88*x^0.89};
    \addplot+[no marks, black, domain=1:32] {x};
    \node[fill=white] at (rel axis cs: 0.075, 0.95) {\sffamily(a)};
\nextgroupplot[
        xmin=45.2548339959,  % approx 32√2
        xmax=5792.61875148,  % approx 4096√2
        xmode=log,
        log basis x=2,
        ymode=log,
        ymin=0.5,
        ymax=128,
        log basis y=2,
        log origin=infty,
        width=0.45\textwidth,
        height=0.45\textwidth,
        axis lines=center,
        xlabel={threads},
        ylabel={speedup},
        xlabel near ticks,
        ylabel near ticks,
        ytick={1, 4, 16, 64},
        legend style={at={(0.5, 0.9)}, anchor=center, draw=none,
                      /tikz/every even column/.append style={column sep=5pt}},
        legend cell align={left},
        legend columns=1
    ]
    
    \addplot+[only marks, mark=diamond*, color=tol/vibrant/magenta, mark options={scale=2, fill=tol/vibrant/magenta}] coordinates {%
        (64  , {1.30763/0.80977})
        (128 , {1.30763/0.43929})
        (256 , {1.30763/0.22434})
        (512 , {1.30763/0.11258})
        (1024, {1.30763/0.05681})
        (2048, {1.30763/0.03074})
        (4096, {1.30763/0.01689})
    };
    \addplot+[only marks, mark=triangle*, color=tol/vibrant/blue, mark options={scale=2, fill=tol/vibrant/blue}] coordinates {%
        (64  , {1.33840/0.96698})
        (128 , {1.33840/0.49359})
        (256 , {1.33840/0.26208})
        (512 , {1.33840/0.14286})
        (1024, {1.33840/0.08252})
        (2048, {1.33840/0.05351})
        (4096, {1.33840/0.03905})
    };
    \addplot+[no marks, dashed, black, domain=64:4096] {1.61*(x/64)^0.93};
    \addplot+[no marks, dotted, black, domain=64:4096] {1.38*(x/64)^0.93};
    \addplot+[no marks, black, domain=64:4096] {1.61*x/64};
    \node[fill=white] at (rel axis cs: 0.075, 0.95) {\sffamily(b)};
\end{groupplot}
\path (unst-strong c1r2.south west|-current bounding box.south)--
coordinate(legendpos) (unst-strong c1r2.south east|-current bounding box.south);
\matrix[
    matrix of nodes,
    anchor=north,
    inner sep=0.2em,
    %draw
  ]at([yshift=-1ex]legendpos)
  {
    \ref{plot:unst-strong-interp}& Algorithm \ref{algo:par-interp}&[5pt]
    \ref{plot:unst-strong-spread}& Algorithm \ref{algo:otf-spread}\\};
\end{tikzpicture}
\caption{%
Strong scaling results for Algorithm \ref{algo:otf-spread} and grid spacing
$h = 0.5\um$ (a grid refinement of 64) for $2^{16}$ randomly placed IB points
in a $16\um\times16\um\times16\um$ triply periodic domain for (a) 1-32 CPU
cores, and (b) 64-4096 threads on the GPU. Speedup is measured relative to
serial Algorithms \ref{algo:par-interp} and \ref{algo:serial-spread}. The solid
black lines show the trendline for ideal speedup. The dashed or dotted lines
give the initial trend for interpolation and spreading, respectively.
}
\label{fig:unstructured-strong}
\end{figure}

Figure \ref{fig:unstructured-strong} shows the results of these tests. Speedup is
measured relative to the serial interpolation and spreading implementations.  The
trendlines estimate that increasing computing resources by a factor of two decreases
runtime by a factor of about 1.91 for CPU and GPU interpolation and by a factor of about
1.85 for CPU spreading. It is not trivial to limit the number of threads used by
{\thrust} for work done on the GPU, so the key-value sort and segmented reduce use as
many threads as {\thrust} decides is prudent.  While the trendline indicates a decrease
in runtime by a factor of 1.91 as well, this is merely an approximation.  

Parallel CPU interpolation using a single processor is identical to the serial
CPU interpolate, so the CPU interpolate passes through $(1,\,1)$. The same is
not true of parallel spreading using a single processor compared to serial
spreading. Because of the additional sort step in the parallel spreading
algorithm, single-threaded Algorithm \ref{algo:par-spread} is about 12\% slower
than its serial counterpart. The CPU code also enjoys the benefit of using
vector registers for some of the computation. The GPU requires 64 threads to
match the speed of a single CPU core.
%Based on the trendline in Figure
%\ref{fig:unstructured-strong} and the speedup for grid refinement shown in
%Figure \ref{fig:grid-dependence}, we see that the GPU attains a speedup of
%approximately 5.25$\times$ compared to 32 CPU processors, and uses
%approximately 4480 threads, where the kernel reaches the limit on register
%memory.
%
Even at 4096 threads, interpolate on the GPU shows no indication of plateauing. The final
data point for that curve shows a speedup of approximately $77\times$. Figure
\ref{fig:grid-dependence}(b), on the other hand, shows that the maximum speedup we can
expect for this problem is approximately 85$\times$, which the trendline in Figure
\ref{fig:unstructured-strong}(b) predicts will occur at approximately 4480 threads. Thus,
we can expect the plateau for interpolate on the GPU to be very abrupt. This indicates a
hardware limitation, the likely culprit being exhaustion of register memory. The
plateauing of the CPU curves is not a limitation of the algorithm for the CPU. Despite
having 48 cores, the test using 32 cores did not utilize them at full capacity. Using
fewer cores, on the other hand, was able to maintain full utilization for the duration of
the test. If not for having a comparatively limited number of CPU cores, we expect to see
the CPU trend continue.

If not for hardware limitations, it seems that the algorithm scales without bound on
either the CPU of GPU. Overall, trends for the CPU and GPU are very similar. Because of
these similarities, we will restrict ourselves to the GPU, but expect any conclusions to
hold for the CPU as well.

\subsubsection{Weak scaling}\label{sec:unst-weak}
In contrast, with improved computing resources, we may wish to solve bigger problems. The
ideal parallel algorithm solves a problem with $p$ threads in the same time as it solves
a twice bigger problem with $2p$ threads. Here, we place between $2^{16}$ and $2^{19}$
points randomly in the domain. We increase the number of threads proportionally, between
128 and 1024.  

\begin{table}[ht]
    \begin{center}
        \begingroup
        \setlength{\tabcolsep}{9pt}
        \renewcommand{\arraystretch}{1.5}
%        \begin{tabular}{ccccc}
%                                                                                               \toprule
%            $p$  & $n$      & \titletable{interpolate}{20000}  & \titletable{spread}{10000} \\ \midrule
%            128  & $2^{16}$ & $0.43930 \pm 0.00372 $           & $0.47632 \pm 0.00278 $     \\
%            256  & $2^{17}$ & $0.44918 \pm 0.01097 $           & $0.46503 \pm 0.00050 $     \\
%            512  & $2^{18}$ & $0.45072 \pm 0.01195 $           & $0.44533 \pm 0.00054 $     \\
%            1024 & $2^{19}$ & $0.45442 \pm 0.00960 $           & $0.43561 \pm 0.00047 $     \\ \bottomrule
%        \end{tabular}
        \begin{tabular}{ccccc}
                                                                                               \toprule
            $p$  & $n$      & \titletable{interpolate}{20000} & \titletable{spread}{10000} \\ \midrule
            128  & $2^{16}$ & $0.43930$                       & $0.47632$                  \\
            256  & $2^{17}$ & $0.44918$                       & $0.46503$                  \\
            512  & $2^{18}$ & $0.45072$                       & $0.44533$                  \\
            1024 & $2^{19}$ & $0.45442$                       & $0.43561$                  \\ \bottomrule
        \end{tabular}                                                                                             \endgroup
    \end{center}
    \caption{%
Weak scaling results for interpolation and spreading for $p$ threads and $n$ randomly
placed IB points in a $16\um\times16\um\times16\um$ triply periodic domain with
$h=0.5\um$ on the GPU. 
%95\% confidence intervals for average time per call are reported
Average time per call is reported
in seconds. $N$ is the number of samples taken.
    }
    \label{tab:unstructured-weak}
\end{table}

Table \ref{tab:unstructured-weak} lists runtimes for increasing threads and
problem size on the GPU. Interpolate scales nearly perfectly with a difference
of $15\ms$ (\textasciitilde3\%) increase between the problem with 128 threads
and $2^{16}$ IB points and that with 1024 threads and $2^{19}$ points. Spread,
on the other hand, decreases in time as the problem size increases. This
speedup is artificial, and should not be expected in general. In the $n=2^{16}$
case, there is 1 IB point for every 4 grid cells, on average. When $n=2^{19}$,
the density increases to 2 for every grid cell. As a result, it becomes
increasingly unlikely to find a cell containing no IB points. This means that
writing the values to the output vector(s) becomes increasingly coalesced,
which, in turn, reduces the number of writes to global memory and vastly
improves the speed of the write overall. Typical use of the IB method does not
have IB points in every grid cell, but the recommendations that IB points on
connected structures be spaced $0.5h$--$h$ apart typically yields 1--4 IB
points in each occupied grid cell. We now consider a more typical use of the IB
method.

% vim: cc=90 tw=89


\subsection{Elastic Objects}

We are motivated by the desire to simulate the motion of cells immersed in a fluid. Cells
are not randomly generated points, but cohesive structures, kept together by elastic
forces and the near-constant volume enclosed by their membranes. In this section, we
replace the randomly-placed IB points with points sampled on the surface of either a
sphere or an RBC. We track $n_d$ data sites per object, and interpolate fluid velocities
only to these points. We construct an RBF interpolant based on the positions of the data
sites and evaluate forces at $n_s$ sample sites, chosen so that neighboring sample sites
are initially within approximately $0.5h$ of each other. We spread forces from the sample
sites. It is generally the case that $n_s > n_d$, so that we interpolate to fewer points
than we spread from. In the parlance of Section \ref{sec:parallel}, $n_\gamma=n_d$ in the
context of interpolation, and $n_\gamma=n_s$ in the context of spreading. These point
sets are generated by the \texttt{KernelNode} library. In this case, we invoke the fluid
solver, so that as the object deforms, the force it imparts on the fluid will affect the
fluid velocity. The sphere and RBC are elastic, obeying the Skalak constitutive law
\cite{Skalak:1973tp} with shear modulus $2.5\times10^{-3}\dynpercm$ and bulk modulus
$2.5\times10^{-1}\dynpercm$. The RBC has rest configuration given in \cite{Omori:2012hw}:
\begin{equation}
    \begin{aligned}
        x(\theta,\,\varphi) &= R_0\cos\theta\cos\varphi, \\
        y(\theta,\,\varphi) &= R_0\sin\theta\cos\varphi, \\
        z(\theta,\,\varphi) &= \sfrac12 R_0\sin\varphi\left[0.21 + 2.0 \cos^2\varphi -1.12 \cos^4\varphi\right],
    \end{aligned}
\end{equation}
where $R_0=3.91\um$, $\theta\in[-\pi,\,\pi)$, and $\varphi\in[-\pi/2,\,\pi/2]$. These
tests require a timestep of $k=0.1\us$ for stability. With $\shear=1000\si{\per\second}$,
an IB point requires at least 32 timesteps to transit a grid cell, so unlike the tests
using randomly placed IB points above, there will be considerably more redundant
computation. We first validate the fluid solver with the elastic sphere before performing
scaling tests, similar to those above, with RBCs.

\subsubsection{Convergence study}
To test the convergence of the fluid solver and cell representation, consider 
an object that obeys Skalak's law with the coefficients given above and is
spherical at rest. Deform the object from its rest configuration by stretching
it by a factor of $1.1$ in the $z$ direction and compressing it by a factor of
$1.1$ in the $y$ direction to maintain a fixed volume. For this test, $\shear$
is zero, so the fluid velocity is intially zero, and the object tends toward
its rest configuration over the course of simulation. We allow the object to
relax for $16\us$, and compare errors generated by successive grid refinements
of 16, 32, 64, and 128 points per $16\um$. For each grid, we use a fixed set of
$n_d=625$ data sites, sampled approximately uniformly on the surface of the
sphere, and choose $n_s$ so that sample sites are approximately uniform and
roughly $h/1.1$ apart, so that sample sites are roughly $h$ apart initially.
For $h=1\um$, $n_s=220$, and a refinement by a factor of 2 increases $n_s$ by a
factor of 4, for a maximum number of 14080 sample sites for these tests. A thin
interface which generates a force will cause a jump in the normal stress across
the interface, which the IB method may not recover. We therefore anticipate
first order convergence for the fluid velocity and data site positions.

\begin{table}
    \begin{center}
        \begingroup
        \setlength{\tabcolsep}{9pt}
        \renewcommand{\arraystretch}{1.5}
        \begin{tabular}{cc|cc|cc}
                                                                                                                 \\ \toprule
            $h$       & $k$       & $\|\u_h-\u_{0.5h}\|_2$ & order    & $\|\u_h-\u_{0.5h}\|_{\infty}$ & order    \\ \midrule
            $1.00\um$ & $1.6\us$ & $2.22736\times10^{-2}$ &          & $7.41870\times10^{-2}$        &          \\
            $0.50\um$ & $0.4\us$ & $4.32802\times10^{-4}$ & 5.715132 & $1.90832\times10^{-3}$        & 5.280787 \\
            $0.25\um$ & $0.1\us$ & $1.46835\times10^{-4}$ & 1.559507 & $1.08472\times10^{-3}$        & 0.814971 \\ \bottomrule
        \end{tabular}
        \endgroup
    \end{center}
    \caption{%
Convergence of the fluid velocity for a deformed sphere returning to its rest
configuration in a $16\um\times16\um\times16\um$ triply periodic domain at
$t=160\us$. The finest grid, with $h=0.125\um$ uses timestep $k=0.025\us$.
    }
    \label{tab:u-convergence}
\end{table}

\begin{table}
    \begin{center}
        \begingroup
        \setlength{\tabcolsep}{9pt}
        \renewcommand{\arraystretch}{1.5}
        \begin{tabular}{cc|cc|cc}
                                                                                                             \\ \toprule
            $h$       & $n_s$ & $\|\X_h-\X_{0.5h}\|_2$ & order    & $\|\X_h-\X_{0.5h}\|_{\infty}$ & order    \\ \midrule
            $1.00\um$ & 220   & $2.96106\times10^{-3}$ &          & $4.78122\times10^{-3}$        &          \\
            $0.50\um$ & 880   & $7.29973\times10^{-4}$ & 2.020201 & $1.16865\times10^{-3}$        & 2.032531 \\
            $0.25\um$ & 3520  & $3.59088\times10^{-4}$ & 1.023507 & $6.09555\times10^{-4}$        & 0.939021 \\ \bottomrule
        \end{tabular}
        \endgroup
    \end{center}
    \caption{%
Convergence of data sites for a deformed sphere returning to its rest
configuration in a $16\um\times16\um\times16\um$ triply periodic domain at
$t=16\us$. For each grid, we track 625 data sites on the sphere. The finest
grid, with $h = 0.125\um$ used $n_s=14080$ sample sites.
    }
    \label{tab:x-convergence}
\end{table}

Tables \ref{tab:u-convergence} and \ref{tab:x-convergence} show the convergence
of fluid velocity and data sites, respectively. To compute errors in the fluid
velocity, we construct a cubic spline from the velocities on the finer grid
and evaluate the spline at the grid points of the coarser grid. Each of the
interfaces is constructed with the same number of data sites, so the
coordinates from one simulation to another can be compared directly. We recover
approximately first order convergence for both fluid velocity and data sites
positions. Having established asymptotic convergence of our IB solver, we
continue by performing scaling tests with RBCs.

\subsubsection{Strong scaling}

We again wish to see how these algorithms can help speed up the runtime of
a fixed problem. Here, we consider tests with a single RBC and with 4 RBCs.
To construct the RBCs, we now use $n_d=864$ data sites, for an initial data
site spacing of approximately $1.6h$, and $n_s=8832$ sample sites per cell, for
an initial sample site spacing of approximately $0.5h$. We use a timestep of
$k=0.1\us$ to simulate the motion of the cells for $1\ms$.

%\begin{table}
%    \begin{center}
%        \begingroup
%        \setlength{\tabcolsep}{9pt}
%        \renewcommand{\arraystretch}{1.5}
%        \begin{tabular}{ccccc}
%            $p$  & cells & \titletable{interpolate}{20000} & \titletable{spread}{10000} \\ \hline
%%            1    & 1     & $0.47633 \pm 0.00024$           & $4.58460 \pm 0.00821$      \\
%%            2    & 1     & $0.26608 \pm 0.00146$           & $2.59113 \pm 0.00567$      \\
%%            4    & 1     & $0.14505 \pm 0.00080$           & $1.44780 \pm 0.00353$      \\
%%            8    & 1     & $0.07593 \pm 0.00031$           & $0.76037 \pm 0.00229$      \\
%%            16   & 1     & $0.04010 \pm 0.00009$           & $0.40348 \pm 0.00137$      \\
%%            32   & 1     & $0.02037 \pm 0.00007$           & $0.22602 \pm 0.00105$      \\
%            64   & 1     & $0.01080 \pm 0.00004$           & $0.11888 \pm 0.00055$      \\
%            128  & 1     & $0.00586 \pm 0.00002$           & $0.06535 \pm 0.00023$      \\
%            256  & 1     & $0.00341 \pm 0.00002$           & $0.03912 \pm 0.00012$      \\
%            512  & 1     & $0.00179 \pm 0.00001$           & $0.02649 \pm 0.00014$      \\
%            1024 & 1     & $0.00093 \pm 0.00001$           & $0.01924 \pm 0.00011$      \\
%            2048 & 1     & $0.00094 \pm 0.00001$           & $0.01627 \pm 0.00012$      \\
%            4096 & 1     & $0.00093 \pm 0.00001$           & $0.01455 \pm 0.00009$%      \\
%%            8192 & 1     & $0.00094 \pm 0.00001$           & $0.01413 \pm 0.00010$
%        \end{tabular}
%        \endgroup
%    \end{center}
%    \caption{%
%        Results of strong scaling tests for IB spreading and interpolate.
%        Reported values are in seconds.
%    }
%\end{table}

\begin{figure}[ht]
\begin{tikzpicture}
\begin{groupplot}[
    group style={group name=rbc-strong, group size=2 by 1},
    height=0.5\textwidth,
    width=0.5\textwidth
]
\nextgroupplot[
        xmode=log,
        xmin=45.2548339959,
        xmax=5792.61875148,
        log basis x=2,
        ymode=log,
        ymax=64,
        log basis y=2,
        log origin=infty,
        width=0.5\textwidth,
        height=0.5\textwidth,
        axis lines=center,
        xlabel={threads},
        ylabel={speedup},
        xlabel near ticks,
        ylabel near ticks,
        legend style={at={(0.8, 0.2)}, anchor=center},
        legend cell align={left}
]
    \addplot+[only marks, mark=diamond*, color=tol/vibrant/magenta, mark options={fill=tol/vibrant/magenta}] coordinates {%
        (64  , {0.01080/0.01080})
        (128 , {0.01080/0.00586})
        (256 , {0.01080/0.00341})
        (512 , {0.01080/0.00179})
        (1024, {0.01080/0.00093})
        (2048, {0.01080/0.00094})
        (4096, {0.01080/0.00093})
    }; \label{plot:rbc-interp}
    \addplot+[only marks, mark=triangle*, color=tol/vibrant/blue, mark options={fill=tol/vibrant/blue}] coordinates {%
        (64  , {0.11888/0.11888})
        (128 , {0.11888/0.06535})
        (256 , {0.11888/0.03912})
        (512 , {0.11888/0.02649})
        (1024, {0.11888/0.01924})
        (2048, {0.11888/0.01627})
        (4096, {0.11888/0.01455})
    }; \label{plot:rbc-spread}
    \addplot+[no marks, dashed, black, domain=64:4096] {(x/64)^0.9};
    \addplot+[no marks, black, domain=64:4096] {(x/64)};
    \node [fill=white] at (rel axis cs: 0.075, 0.95) {(a)};
\nextgroupplot[
        xmode=log,
        xmin=45.2548339959,
        xmax=5792.61875148,
        log basis x=2,
        ymode=log,
        ymax=64,
        log basis y=2,
        log origin=infty,
        width=0.5\textwidth,
        height=0.5\textwidth,
        axis lines=center,
        xlabel={threads},
        xlabel near ticks,
        ylabel near ticks,
        legend style={at={(0.8, 0.2)}, anchor=center},
        legend cell align={left}
    ]
    \addplot+[only marks, mark=diamond*, color=tol/vibrant/magenta, mark options={fill=tol/vibrant/magenta}] coordinates {%
        (64  , {0.04150/0.04150})
        (128 , {0.04150/0.02251})
        (256 , {0.04150/0.01172})
        (512 , {0.04150/0.00603})
        (1024, {0.04150/0.00349})
        (2048, {0.04150/0.00181})
        (4096, {0.04150/0.00103})
    };
    \addplot+[only marks, mark=triangle*, color=tol/vibrant/blue, mark options={fill=tol/vibrant/blue}] coordinates {%
        (64  , {0.40148/0.40148})
        (128 , {0.40148/0.20628})
        (256 , {0.40148/0.11208})
        (512 , {0.40148/0.06313})
        (1024, {0.40148/0.03931})
        (2048, {0.40148/0.02785})
        (4096, {0.40148/0.02168})
    };
    \addplot+[no marks, dashed, black, domain=64:4096] {(x/64)^0.9};
    \addplot+[no marks, black, domain=64:4096] {(x/64)};
    \node [fill=white] at (rel axis cs: 0.075, 0.95) {(b)};
\end{groupplot}
\path (rbc-strong c1r1.south west|-current bounding box.south)--
coordinate(legendpos) (rbc-strong c2r1.south east|-current bounding box.south);
\matrix[
    matrix of nodes,
    anchor=north,
    inner sep=0.2em,
    %draw
  ]at([yshift=-1ex]legendpos)
  {
      \ref{plot:rbc-interp}& Algorithm \ref{algo:par-interp}&[5pt]
      \ref{plot:rbc-spread}& Algorithm \ref{algo:otf-spread} \\};
\end{tikzpicture}
\caption{%
    Speedup of Algorithms \ref{algo:par-interp} and \ref{algo:otf-spread} with
    increasing numbers of threads compared to 64 threads on the GPU for (a) 1
    and (b) 4 RBCs. Speedup is measured relative to the time taken for each
    algorithm using 64 threads on the GPU. Dashed lines indicate trends, and
    solid lines indicate ideal scaling.
}
\label{fig:str-strong}
\end{figure}

Figure \ref{fig:str-strong} shows the speedup observed with increasing threads
for 1 and 4 RBCs for 64--4096 threads on the GPU. We again see that the initial
speedup for interpolation is nearly linear with increased threads. In subfigure
\ref{fig:str-strong}(a), there is a sharp plateau that corresponds to every
data site having its own thread. In other words, there are more threads than
there is work to be done, since we track only 864 data sites for a single RBC.
Subfigure \ref{fig:str-strong}(b), on the other hand, has 3456 data sites, so
the trend continues for 512--4096 threads. In this case, we expect this graph
to plateau beyond 4096, when each data site has its own thread. However, we do
not expect for the trend to continue with more cells, as the presumed maximum
number of threads for interpolation is 4480, as discussed in Section
\ref{sec:unst-strong}. Comparing subfigure \ref{fig:str-strong}(a)
to (b), we see that the speedup in spreading is also dependent on the amount of
work. This indicates that, as with interpolation, the maximum speedup for
spreading is limited by hardware, rather than being a limitation of the
algorithm.

The trendlines for these tests indicate that increasing computing resources
by a factor of two decreases runtime by a factor of about 1.87 for these
algorithms. Again, this is merely an approximation as the sort and reduction
steps of the spreading algorithm are provided by {\thrust}, and therefore are
not limited to the listed number of threads. The similarity between the result
of the tests with RBCs and with randomly placed points indicates that the
distribution of points does not have a marked impact on the efficacy of the
parallelization for a fixed problem. We now see if the same holds for weak
scaling tests.


%\begin{table}
%    \begin{center}
%        \begingroup
%        \setlength{\tabcolsep}{9pt}
%        \renewcommand{\arraystretch}{1.5}
%        \begin{tabular}{ccccc}
%            $p$   & cells & \titletable{interpolate}{20000} & \titletable{spread}{10000} \\ \hline
%            64    & 4     & $0.04150 \pm 0.00013$           & $0.40148 \pm 0.00146$      \\
%            128   & 4     & $0.02251 \pm 0.00003$           & $0.20628 \pm 0.00060$      \\
%            256   & 4     & $0.01172 \pm 0.00002$           & $0.11208 \pm 0.00031$      \\
%            512   & 4     & $0.00603 \pm 0.00002$           & $0.06313 \pm 0.00014$      \\
%            1024  & 4     & $0.00349 \pm 0.00002$           & $0.03931 \pm 0.00011$      \\
%            2048  & 4     & $0.00181 \pm 0.00001$           & $0.02785 \pm 0.00011$      \\
%            4096  & 4     & $0.00103 \pm 0.00001$           & $0.02168 \pm 0.00009$
%        \end{tabular}
%        \endgroup
%    \end{center}
%    \caption{%
%        Results of strong scaling tests for IB spreading and interpolate with
%        4 RBCs.
%        Reported values are in seconds.
%    }
%\end{table}

\subsection{Weak scaling of interaction operations}

\begin{table}
    \begin{center}
        \begingroup
        \setlength{\tabcolsep}{9pt}
        \renewcommand{\arraystretch}{1.5}
        \begin{tabular}{ccccc}
            $p$ & cells & \titletable{interpolate}{20000} & \titletable{forces}{10000} & \titletable{spread}{10000} \\ \hline
            64  & 1     & $0.01079 \pm 0.00004$           & $0.00242 \pm 0.00002$      & $0.11881 \pm 0.00055$      \\
            128 & 2     & $0.01165 \pm 0.00003$           & $0.00249 \pm 0.00002$      & $0.11219 \pm 0.00051$      \\
            256 & 4     & $0.01171 \pm 0.00003$           & $0.00287 \pm 0.00003$      & $0.11214 \pm 0.00036$      \\
            512 & 8     & $0.01199 \pm 0.00003$           & $0.00532 \pm 0.00008$      & $0.11354 \pm 0.00047$
        \end{tabular}
        \endgroup
    \end{center}
    \caption{%
        Weak scaling results.
    }
\end{table}

\begin{table}
    \begin{center}
        \begingroup
        \setlength{\tabcolsep}{9pt}
        \renewcommand{\arraystretch}{1.5}
        \begin{tabular}{ccccc}
            $p$  & $n$      & \titletable{interpolate}{20000} & \titletable{forces}{10000} & \titletable{spread}{10000} \\ \hline
            128  & $2^{16}$ & $0.43930 \pm 0.00019$           & $0.00026 \pm 0.00031$      & $0.47632 \pm 0.00142$      \\
            256  & $2^{17}$ & $0.44918 \pm 0.00056$           & $0.00024 \pm 0.00002$      & $0.46503 \pm 0.00026$      \\
            512  & $2^{18}$ & $0.45072 \pm 0.00061$           & $0.00029 \pm 0.00001$      & $0.44533 \pm 0.00028$      \\
            1024 & $2^{19}$ & $0.45442 \pm 0.00049$           & $0.00055 \pm 0.00000$      & $0.43561 \pm 0.00024$      \\
        \end{tabular}
        \endgroup
    \end{center}
    \caption{%
        Results of $n$ randomly placed IB points in a 16 \si{\micro\meter}
        $\times$ 16 \si{\micro\meter} $\times$ 16 \si{\micro\meter} periodic
        domain.
    }
\end{table}


% vim: cc=90 tw=89
