\subsection{Spatial discretization}

The equations [@eq:...] are discretized on staggered, regular grids with grid
spacing $h$. The components of vector-valued quantities are discretized on the
center of the corresponding cell face. That is, for example, if
$\vec{a} = (a_1,\,a_2,\,a_3)$ are the three components of a vector-valued
quantity for some grid cell with lower corner $h(i,\,j,\,k)$, then $a_1$
lives at $h(i,\,j+0.5,\,k+0.5)$, $a_2$ at $h(i+0.5,\,j,\,k+0.5)$, and $a_3$ at
$h(i+0.5,\,j+0.5,\,k)$. Scalar quantities are discretized at cell centers,
e.g., $h(i+0.5,\,j+0.5,\,k+0.5)$.

The central difference operator $D_i$, which differentiates with respect to
the $i^\text{th}$ coordinate, in the absence of solid boundaries, is defined as
\begin{equation}
    D_ig(\vec{x}) = \frac{g(\vec{x}+h\vec{e}_i/2) + g(\vec{x}+h\vec{e}_i/2)}{h}.
\end{equation}
In the presence of solid boundaries, a linear interpolant is used to fill a
ghost value. The divergence operator applied to a vector-valued function
$\vec{a}$ on the staggered grid, $\grad_h\cdot\vec{a} = D_ia^i$, where Einstein
summation notation is assumed and $i$ ranges from 1 to $d$, results in
a scalar-valued quantity at the cell center. The gradient operator applied to
scalar-valued quantity $b$ at the cell center yields the vector
$\grad_hb = \vec{e}^jD_jb$ on the staggered grid.

Define the central averaging operator, $A^i$, which averages in the
$i^\text{th}$ coordinate direction and in the absence of boundaries, as
\begin{equation}
    A_i = \frac{g(\vec{x}+h\vec{e}_i)+g(\vec{x}-h\vec{e}_i)}{2}.
\end{equation}
In the presence of boundaries, we use the value at the ghost cell. By taking
products of averaged components of $\vec{u}$, we compute advection in
conservation form as
\begin{equation}
    \vec{H}(\vec{u}) := \vec{e}_iD_j((\delta^{im}A_mu^j)(\delta^{jn}A_nu^i)).
\end{equation}

The viscous terms are computed as $\mu\Delta_h\vec{u}$, where $\Delta_h$ is the
standard 5- or 7-point Laplacian. In the absence of solid boundaries, this
is identical to  $\Delta_h = \grad_h\cdot\grad_h$. Otherwise, we use the value
at the ghost cell for stencils that extend past a solid boundary.

\subsection{Temporal discretization}

We time-step using the 2$^\text{nd}$ order implicit-explicit Runge-Kutta method
as described in [@Peskin:...] and projection method PmIII from [@Brown:...].
This leads to the system of equations
\begin{align}
    \left(\tilde{I}-\frac{\lambda}2\Delta_h\right)\delta\vec{w}
    &= \alpha\lambda\left[\Delta_h\vec{u}^n + B_h \vec{w}_b^{n+\sfrac12}\right] + \tilde{I}\left(\rho^{-1}\vec{f} - \vec{H}(\vec{u}^{n+\alpha-\sfrac12})\right) \\
    \vec{w} &= \vec{u}^n + \delta\vec{w} \\
    k\Delta_h\phi &= \grad_h\cdot\vec{w} \\
    \vec{u}^{n+\alpha} &= \vec{w} - k\grad_h\phi,
\end{align}
where $k$ is the timestep, $\lambda = k\mu/\rho$, $B_h$ is an appropriate
boundary operator for the discrete Laplacian, $\vec{w}_b$ are boundary data for
$\vec{w}$, and $\alpha$ is the fraction of the timestep for the current stage
of the timestepping method: $\alpha=\sfrac12$ or 1. The quantity $\phi$ acts as
a pseudo-pressure, and the pressure can be recovered via
$p=\phi-(\mu/\rho)\grad_h\cdot\vec{w}$.

The modified identity matrix, $\tilde{I}$, corrects errors in using the
standard discrete Laplacian on a staggered grid, where the stencil may extend
beyond the boundary of the domain. This requires a ghost point, to which we
have assigned a value using a linear interpolant. Under certain circumstances,
this works perfectly fine, but in general, the resulting approximation to the
Laplacian is 0$^\text{th}$ order, but differs by a constant factor. Let that
factor be denoted $1-\epsilon_h$. For stencils that straddle the boundary, we
modify the momentum equation so that we are instead solving
\begin{equation}
    (1-\epsilon_h)\rho(\vec{u}_t + \grad\cdot(\vec{u}\otimes\vec{u})) = (1-\epsilon_h)\left[\mu\Delta\vec{u} + \vec{f}\right].
\end{equation}
This requires no additional modification to the Laplacian, but after
discretization, yields a factor in front of the intermediate velocity update
$\delta\vec{w}$ and the force density and advection terms. Under reasonable
assumptions, the resulting linear system is symmetric positive definite, and
is suitable for use with preconditioned conjugate gradients. One can view this
as using a quadratic interpolant near the boundaries and scaling the
corresponding rows to ensure symmetry of the linear system. The resulting
systems are identical.
