\subsection{Serial spread}

\begin{algorithm}
\caption{Serial spread}
\label{algo:serial-spread}
\begin{algorithmic}[1]
\Procedure{serial-spread}{$\interface_h,\,\domain_h,\,\vec{L}$}
\State $\triangleright\ $\textbf{generate}: Values of $\ell$ sampled at each point in $\domain_h$
\For {$i = 1,\,\ldots,\,n_\gamma$}\Comment{Loop over IB points}
    \State $\x \gets h(\idx{\X_i}+\stag)$\Comment{$\X_i\in\interface_h$, $\x\in\domain_h$}\label{line:serial-spread-x}
    \State $\Delta\x \gets \x-\X_i$
    \For {$j = 1,\,\ldots,\,s^d$}\Comment{Loop over shifts}
        \State $w \gets \Dirac_h(\Delta\x+h\shift_j)$
        \State $k \gets \#(\idx{\x} + \shift_j)$ \label{line:serial-spread-k}
        \If {$k\neq\error$}
            \State $\ell_k \gets \ell_k + w \cdot L_i$ \label{line:serial-spread-write}
        \EndIf
    \EndFor
\EndFor
\State \Return $\vec{\ell}$
\EndProcedure
\end{algorithmic}
\end{algorithm}

Algorithm \ref{algo:serial-spread} lists an example serial implementation of spreading in
pseudocode. The overall structure consists of two loops: a loop over the (indices of)
IB points, and a loop over the (indices of) shifts. From this, we see that for a fixed
choice of $\kernel$, and therefore $s$, the amount of work is $\bigo{n_\gamma}$, i.e.,
independent of the Eulerian grid. The spread values are accumulated into a vector
$\vec{\ell}$ (line \ref{line:serial-spread-write}). The target index, $k$ (line
\ref{line:serial-spread-k}), is computed using the grid indexing function, introduced in
the previous section. Of note is that $k$ depends on $\shift_j$, and $\x$, which in turn
depends on $\X_i$, as seen on line \ref{line:serial-spread-x}. This means that $k$
depends on both loop indices. There is no guarantee that unique pairs of the loop
variables $i$ and $j$ will yield unique target indices. As a result, simply parallelizing
one or both of the loops may lead to write contentions.

The property that runtime be independent of the Eulerian grid is desirable, as the
number of IB points is often much fewer than the number of grid points. In other words,
many grid cells will be empty of IB points, and unless they have nearby IB points, there
is no useful work to be done for that grid cell. An algorithm that depends on the
Eulerian grid will invariably involve wasted computational resources. We therefore aim to
preserve the independence property. This is achieved straightforwardly for interpolation.

% vim: cc=90 tw=89
