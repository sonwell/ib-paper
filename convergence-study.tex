\subsubsection{Convergence study}
To test the convergence of the fluid solver and cell representation, we
consider a single RBC in the domain. We choose sets of $n_d$ and $n_s\approx
4n_d$ points, as in [@Shankar:XXX], sampled approximately uniformly on the
surface of the at-rest RBC. We maintain an initial data site spacing of
approximately $h$ under refinement. Refining $h$ by a factor of 2 requires
approximately four times the data sites, so we generate a cascade of potential
point sets with 31, 132, 546, 2210, and 8832 points. Each successive pair from
this list can then be used for the set of data sites and sample sites,
respectively. We simulate the motion of the RBC for $0.4\ms$ and compare the
errors generated by successive grid refinements of 16, 32, 64, and 128 points
per $16\um$. The cell membrane is modeled as being infinitesimally thin, so
forces tangent to the membrane result in a jump in the normal stress across the
membrane. Due to the nature of the IB method, this jump will be smoothed out,
so we anticipate, at best, first order convergence in the fluid velocity in the
max norm, and up to second order in the $L^2$ norm.


\begin{table}
    \begin{center}
        \begingroup
        \setlength{\tabcolsep}{9pt}
        \renewcommand{\arraystretch}{1.5}
        \begin{tabular}{cc|cc|cc}
                                                                                                                                    \\ \toprule
            $h$       & $k$      & $\|\vec{u}_h-\vec{u}_{0.5h}\|_2$ & order    & $\|\vec{u}_h-\vec{u}_{0.5h}\|_{\infty}$ & order    \\ \midrule
            $1.00\um$ & $1.6\us$ & 0.000789                         &          & 0.006200                                &          \\
            $0.50\um$ & $0.4\us$ & 0.000331                         & 1.253194 & 0.004700                                & 0.399607 \\
            $0.25\um$ & $0.1\us$ & 0.000111                         & 1.576272 & 0.002486                                & 0.918834 \\ \bottomrule
        \end{tabular}
        \endgroup
    \end{center}
    \caption{%
        Convergence of the fluid velocity for the case of a single red blood
        cell in a $16\um\times16\um\times16\um$ triply periodic domain and
        shear-like flow with shear rate $\dot{\gamma} = 1000\si{\per\second}$
        at $t=0.4\ms$. The finest grid, with $h=0.125\um$ used timestep
        $k=0.025\us$, $n_d=2210$ data sites, and $n_s=8832$ sample sites. We
        observe approximately first-order convergence.
    }
    \label{tab:u-convergence}
\end{table}

\begin{table}
    \begin{center}
        \begingroup
        \setlength{\tabcolsep}{9pt}
        \renewcommand{\arraystretch}{1.5}
        \begin{tabular}{cc|cc|cc}
                                                                                                                                 \\ \toprule
            $h$       & $n_d$ & $\|\vec{X}_h-\vec{X}_{0.5h}\|_2$ & order    & $\|\vec{X}_h-\vec{X}_{0.5h}\|_{\infty}$ & order    \\ \midrule
            $1.00\um$ & 31    & $7.91287\times10^{-6}$           &          & $1.57796\times10^{-5}$                  &          \\
            $0.50\um$ & 132   & $5.89760\times10^{-6}$           & 0.405772 & $1.22294\times10^{-5}$                  & 0.351832 \\
            $0.25\um$ & 546   & $4.14315\times10^{-6}$           & 0.497374 & $1.04531\times10^{-5}$                  & 0.221083 \\ \bottomrule
        \end{tabular}
        \endgroup
    \end{center}
    \caption{%
        Convergence of data sites for the case of a single red blood cell in a
        $16\um\times16\um\times16\um$ triply periodic domain and shear-like
        flow with shear rate $\dot{\gamma} = 1000\si{\per\second}$ at
        $t=0.4\ms$. The finest grid, with $h = 0.125\um$ used timestep
        $k=0.025\us$, $n_d=2210$ data sites, and $n_s=8832$ sample sites.
    }
    \label{tab:x-convergence}
\end{table}

Tables \ref{tab:u-convergence} and \ref{tab:x-convergence} show the convergence
of fluid velocity and data sites, respectively. While we recover first-order
convergence of the fluid velocity, the \textcolor{red}{I don't know.}
