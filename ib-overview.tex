\section{Review of the immersed boundary method} \label{sec:ib}

Consider a $d$-dimensional ($d=2$ or 3) rectangular domain $\domain$, which is filled
with a viscous incompressible fluid with constant viscosity $\mu$ and density $\rho$, and
contains an immersed elastic structure, $\interface$. The structure is impermeable to the
fluid and moves at the local fluid velocity, is deformed by this motion, and imparts a
force to the fluid. Otherwise, the interface is treated as part of the fluid.

The fluid velocity, $\u = \u(\x,\,t)$, and pressure, $p = p(\x,\,t)$, are governed by the
incompressible Navier-Stokes equations for a Newtonian fluid,
\begin{gather}
    \label{eq:ins-evol}
    \rho(\u_t + \u\cdot\grad\u) = \mu\Delta\u - \grad p + \f, \\
    \label{eq:ins-incomp}
    \div\u = 0,
\end{gather}
where $\f$ is the elastic force density. This is a set of $d+1$ equations in $d+1$
unknowns: the $d$ components of $\u$, and $p$. The equations are written relative to the
Eulerian frame, so that the coordinates $\x$ are independent variables. Throughout, we
write Eulerian quantities in the lowercase Latin alphabet.

Let $\X=\X(\params,\,t)$ represent a parametrization of the Cartesian coordinates of the
immersed interface with material coordinates $\params$ at time $t$. Let $\L[\X]$ be the
energy density functional for the elastic interface material. The elastic force density
is computed by evaluating% the Fréchet derivative of $\L$,
\begin{equation}
    \label{eq:forces}
    \F = -\delta \L[\X],
\end{equation}
where $\delta$ represents the first variation. Uppercase Latin letters represent
Lagrangian quantities and are functions of $\params$ and $t$.

The interface moves at the local fluid velocity, and force balance on the interface
between the interface and fluid dictates that the interface force on the fluid is equal
to the elastic force. Analytically, the fluid-interface interactions can be written
\begin{gather}
    \label{eq:interpolation}
    \U(\params,\,t) = \int_\domain \Dirac(\x-\X(\params,\,t)) \u(\x,\,t)\d\x,\ \text{and} \\
    \label{eq:spreading}
    \f(\x,\,t) = \int_\interface \Dirac(\x-\X(\params,\,t))\F(\params,\,t)\d\params,
\end{gather}
where $\U(\params,\,t)$ represents the derivative of $\X(\params,\,t)$ with respect to
$t$, and $\Dirac(\x-\X(\params,\,t))$ is the Dirac $\Dirac$-function centered at
$\X(\params,\,t)$. Equation~\eqref{eq:interpolation} is called \term{interpolation}, and
the result of the right-hand side is the fluid velocity at $\X(\params,\,t)$; namely,
$\U(\params,\,t) = \u(\X(\params,\,t),\,t)$. Equation~\eqref{eq:spreading} is called
\term{spreading}, because while $\F(\params,\,t)$ has units of force per unit \emph{area}
on $\interface$, $\f(\x,\,t)$ has units of force per unit \emph{volume} in $\domain$. The
force $\F(\params,\,t)\d\params$ over area $\d\params$ is ``spread'' to the force
$\f(\x,\,t)\d\x$ over volume $\d\x$. 

The fluid equations~\eqref{eq:ins-evol}--\eqref{eq:ins-incomp} are spatially discretized
on a regular background grid of spacing $h$ so that $\domain$ is divided into square or
cubic cells of side length $h$. Because of the checkerboard instability (see, e.g.,~%
\cite{Wesseling:2001ci}) in solving the Navier-Stokes equations on collocated regular
grids, we stagger the components of Eulerian vector quantities. As a result, different
components are discretized at different locations, but corresponding components are
discretized on the same grid. Grid cells for these staggered grids do not coincide. We
therefore consider each of these grids individually. We use $\domain_h$ to denote the set
of grid points for the grid under consideration, and $n_\omega$ to denote the number of
grid points. The momentum equation~\eqref{eq:ins-evol} is discretized in time using a
second-order implicit-explicit Runge-Kutta method as described in~\cite{Peskin:2002go},
and the incompressibility condition~\eqref{eq:ins-incomp} is satisfied using PmIII of~%
\cite{Brown:2001bq}.

The Lagrangian force density~\eqref{eq:forces} is evaluated at a set of points, usually a
\emph{fixed} set of points in material coordinates, referred to as IB points. The
notation $\X_j=\X(\params_j,\,t)$ refers to an individual IB point on $\interface$ in
Cartesian coordinates. The typical heuristic for distributing the points $\X_j$ on a
connected interface places neighboring IB points at most $h$ apart from one another, and
often at most $h/2$ apart. We therefore denote the set of IB points by $\interface_h$ and
the number of IB points by $n_\gamma$. From these points, we construct a smooth
approximation to the interface using radial basis functions (RBFs), as in~%
\cite{Shankar:2015km}. This allows us to calculate geometric properties using well-known
formulas, and evaluate the forces~\eqref{eq:forces} in analytic form, similar to that of%
~\cite{Maxian:2018ek}.

The singular integrals~\eqref{eq:interpolation} and~\eqref{eq:spreading} do not lend
themselves easily to evaluation. In particular, it is unlikely that IB points and grid
points will coincide. For a regular grid with spacing $h$, we replace the Dirac
$\Dirac$-function with a regularized kernel, $\Dirac_h$, which is a product of
one-dimensional kernels, $h^{-1}\kernel(h^{-1}x)$.~\cite{Griffith:2020hi} gives several
options for $\kernel$, but we will restrict ourselves to the simple cosine kernel from~%
\cite{Peskin:2002go}.

Let $\domain_h$ be one of the Eulerian grids for vector-valued quantities. Discretizing
equations \eqref{eq:scalar-interp} and \eqref{eq:scalar-spread} on $\domain_h$ and
$\interface_h$, respectively, yields
\begin{align}
    \label{eq:disc-interp}
    \dot{X}^n_j &= \sum_i \delta_h(\x_i-\X^n_j)u^n_i h^d \quad \text{and} \\
    \label{eq:disc-spread}
    f^n_i &= \sum_j \delta_h(\x_i-\X^n_j)F^n_j \d\theta_j,
\end{align}
where $u^n_i$ and $f^n_i$ are discrete approximations to the components of $\u$ and $\f$
at $\x_i\in\domain_h$ and time $t=t_n$, and $\dot{X}^n_{\smash j}$ and $F^n_{\smash j}$
are their Lagrangian counterparts at $\X_{\smash j}$, respectively. The terms $h^d$ and
$\d\theta_{\smash j}$ are integration weights analogous to $\d\x$ and $\d\params$. For
a topologically spherical interface, we compute weights on the unit sphere using RBFs,
using a simplified version of the method described in~\cite{Fuselier:2013coba}, and use
the Jacobian of the mapping between the sphere and interface to obtain
$\d\theta_{\smash j}$. Both~\eqref{eq:disc-interp} and~\eqref{eq:disc-spread} look like a
matrix-vector multiplication, so we define the \emph{spreading matrix}
$\spread=(\delta_h(\x_i-\X^n_{\smash j}))$ with row $i$ and column $j$. We call its
transpose, $\interp$, the \emph{interpolation matrix}. Collecting values
$\e\cdot\u^n_ih^d$ at each grid point as $\vec{v}^n$ and
$\e\cdot\F^n_{\smash j}\d\theta_{\smash j}$ at each IB point as $\vec{G}^n$, we
rewrite equations~\eqref{eq:disc-interp} and~\eqref{eq:disc-spread} in matrix form as
\begin{align}
    \label{eq:matrix-interp}%
    \U^n &= \interp{\vec{v}}^n \quad\text{and} \\
    \label{eq:matrix-spread}%
    \f^n &= \spread{\vec{G}}^n,
\end{align}
respectively.

Collectively, the equations~\eqref{eq:ins-evol}--\eqref{eq:spreading} constitute the
IB method, and while different implementations depend on myriad details, a single step
proceeds with timestep $k$ roughly as follows:
\begin{enumerate}[label=(\texttt{\alph*})]
    \item interpolate $\u^n$ to $\X^n$ to get $\U^\ast$,
    \item predict Lagrangian positions $\X^\ast = \X^n + k\U^\ast$,
    \item compute Lagrangian forces $\F^\ast$ using positions $\X^\ast$,
    \item spread $-\F^\ast$ from $\X^\ast$ to get $\f^\ast$,
    \item solve for updated Eulerian velocities $\u^\ast$,
    \item project $\u^\ast$ into space of divergence-free vector fields to get
        $\u^{n+1}$,
    \item interpolate $\u^{n+1}$ to $\X^n$ to get $\U^{n+1}$, and
    \item update $\X^{n+1} = \X^n + k \U^{n+1}$.
\end{enumerate}
We group these steps into 3 categories: the purely Eulerian (\texttt{e}) and
(\texttt{f}); the purely Lagrangian (\texttt{b}), (\texttt{c}), and (\texttt{h}); and the
Eulerian-Lagrangian coupling (\texttt{a}), (\texttt{d}), and (\texttt{g}). We discuss the
first two categories in a forthcoming paper. The remainder of this paper discusses the
implementation and parallelization of the third.

% vim: cc=90 tw=89
