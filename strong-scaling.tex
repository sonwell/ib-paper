\subsubsection{Strong scaling}

We again wish to see how these algorithms can help speed up the runtime of
a fixed problem. Here, we consider tests with a single RBC and with 4 RBCs.
To construct the RBCs, we now use $n_d=864$ data sites, for an initial data
site spacing of approximately $0.8h$, and $n_s=8832$ sample sites per cell, for
an initial sample site spacing for approximately $0.25h$. We use a timestep of
$k=0.1\us$ to simulate the motion of the cells for $1\ms$.

%\begin{table}
%    \begin{center}
%        \begingroup
%        \setlength{\tabcolsep}{9pt}
%        \renewcommand{\arraystretch}{1.5}
%        \begin{tabular}{ccccc}
%            $p$  & cells & \titletable{interpolate}{20000} & \titletable{spread}{10000} \\ \hline
%%            1    & 1     & $0.47633 \pm 0.00024$           & $4.58460 \pm 0.00821$      \\
%%            2    & 1     & $0.26608 \pm 0.00146$           & $2.59113 \pm 0.00567$      \\
%%            4    & 1     & $0.14505 \pm 0.00080$           & $1.44780 \pm 0.00353$      \\
%%            8    & 1     & $0.07593 \pm 0.00031$           & $0.76037 \pm 0.00229$      \\
%%            16   & 1     & $0.04010 \pm 0.00009$           & $0.40348 \pm 0.00137$      \\
%%            32   & 1     & $0.02037 \pm 0.00007$           & $0.22602 \pm 0.00105$      \\
%            64   & 1     & $0.01080 \pm 0.00004$           & $0.11888 \pm 0.00055$      \\
%            128  & 1     & $0.00586 \pm 0.00002$           & $0.06535 \pm 0.00023$      \\
%            256  & 1     & $0.00341 \pm 0.00002$           & $0.03912 \pm 0.00012$      \\
%            512  & 1     & $0.00179 \pm 0.00001$           & $0.02649 \pm 0.00014$      \\
%            1024 & 1     & $0.00093 \pm 0.00001$           & $0.01924 \pm 0.00011$      \\
%            2048 & 1     & $0.00094 \pm 0.00001$           & $0.01627 \pm 0.00012$      \\
%            4096 & 1     & $0.00093 \pm 0.00001$           & $0.01455 \pm 0.00009$%      \\
%%            8192 & 1     & $0.00094 \pm 0.00001$           & $0.01413 \pm 0.00010$
%        \end{tabular}
%        \endgroup
%    \end{center}
%    \caption{%
%        Results of strong scaling tests for IB spread and interpolate.
%        Reported values are in seconds.
%    }
%\end{table}

\begin{figure}[ht]
\begin{tikzpicture}
\begin{groupplot}[
    group style={group name=rbc-strong, group size=2 by 1},
    height=0.5\textwidth,
    width=0.5\textwidth
]
\nextgroupplot[
        xmode=log,
        xmin=45.2548339959,
        xmax=5792.61875148,
        log basis x=2,
        ymode=log,
        ymax=64,
        log basis y=2,
        log origin=infty,
        width=0.5\textwidth,
        height=0.5\textwidth,
        axis lines=center,
        xlabel={threads},
        ylabel={speedup},
        xlabel near ticks,
        ylabel near ticks,
        legend style={at={(0.8, 0.2)}, anchor=center},
        legend cell align={left}
]
    \addplot+[only marks, mark=diamond*, color=tol/vibrant/magenta, mark options={fill=tol/vibrant/magenta}] coordinates {%
        (64  , {0.01080/0.01080})
        (128 , {0.01080/0.00586})
        (256 , {0.01080/0.00341})
        (512 , {0.01080/0.00179})
        (1024, {0.01080/0.00093})
        (2048, {0.01080/0.00094})
        (4096, {0.01080/0.00093})
    }; \label{plot:rbc-interp}
    \addplot+[only marks, mark=triangle*, color=tol/vibrant/blue, mark options={fill=tol/vibrant/blue}] coordinates {%
        (64  , {0.11888/0.11888})
        (128 , {0.11888/0.06535})
        (256 , {0.11888/0.03912})
        (512 , {0.11888/0.02649})
        (1024, {0.11888/0.01924})
        (2048, {0.11888/0.01627})
        (4096, {0.11888/0.01455})
    }; \label{plot:rbc-spread}
    \addplot+[no marks, dashed, black, domain=64:4096] {(x/64)^0.9};
    \addplot+[no marks, black, domain=64:4096] {(x/64)};
    \node [fill=white] at (rel axis cs: 0.075, 0.95) {(a)};
\nextgroupplot[
        xmode=log,
        xmin=45.2548339959,
        xmax=5792.61875148,
        log basis x=2,
        ymode=log,
        ymax=64,
        log basis y=2,
        log origin=infty,
        width=0.5\textwidth,
        height=0.5\textwidth,
        axis lines=center,
        xlabel={threads},
        xlabel near ticks,
        ylabel near ticks,
        legend style={at={(0.8, 0.2)}, anchor=center},
        legend cell align={left}
    ]
    \addplot+[only marks, mark=diamond*, color=tol/vibrant/magenta, mark options={fill=tol/vibrant/magenta}] coordinates {%
        (64  , {0.04150/0.04150})
        (128 , {0.04150/0.02251})
        (256 , {0.04150/0.01172})
        (512 , {0.04150/0.00603})
        (1024, {0.04150/0.00349})
        (2048, {0.04150/0.00181})
        (4096, {0.04150/0.00103})
    };
    \addplot+[only marks, mark=triangle*, color=tol/vibrant/blue, mark options={fill=tol/vibrant/blue}] coordinates {%
        (64  , {0.40148/0.40148})
        (128 , {0.40148/0.20628})
        (256 , {0.40148/0.11208})
        (512 , {0.40148/0.06313})
        (1024, {0.40148/0.03931})
        (2048, {0.40148/0.02785})
        (4096, {0.40148/0.02168})
    };
    \addplot+[no marks, dashed, black, domain=64:4096] {(x/64)^0.9};
    \addplot+[no marks, black, domain=64:4096] {(x/64)};
    \node [fill=white] at (rel axis cs: 0.075, 0.95) {(b)};
\end{groupplot}
\path (rbc-strong c1r1.south west|-current bounding box.south)--
coordinate(legendpos) (rbc-strong c2r1.south east|-current bounding box.south);
\matrix[
    matrix of nodes,
    anchor=north,
    inner sep=0.2em,
    %draw
  ]at([yshift=-1ex]legendpos)
  {
      \ref{plot:rbc-interp}& Algorithm \ref{algo:par-interp}&[5pt]
      \ref{plot:rbc-spread}& Algorithm \ref{algo:otf-spread} \\};
\end{tikzpicture}
\caption{%
    Speedup of Algorithms \ref{algo:par-interp} and \ref{algo:otf-spread} with
    increasing numbers of threads compared to 64 threads on the GPU for (a) 1
    and (b) 4 RBCs. Speedup is measured relative to the time taken for each
    algorithm using 64 threads on the GPU. Dashed lines indicate trends, and
    solid lines indicate ideal scaling.
}
\label{fig:str-strong}
\end{figure}

Figure \ref{fig:str-strong} shows the speedup observed with increasing threads
for 1 and 4 RBCs for 64--4096 threads on the GPU. We again see that the initial
speedup for interpolation is nearly linear with increased threads. In subfigure
(a), there is a sharp plateau that corresponds to every data site having its
own thread. In other words, there are more threads than there is work to be
done, since we track only 864 data sites for a single RBC. Subfigure (b), on
the other hand, has 3456 data sites, so the trend continues for 512--4096
threads. In this case, we expect this graph to plateau beyond 4096, as each
data site already has its own thread. However, we do not expect for the trend
to continue with more cells, as the presumed maximum number of threads for
interpolation is 4480, as discussed in Section [@sec:grid-dependence]. 
Comparing subfigure (a) to (b), for spread, we see that the speedup is also
dependent on the amount of work. This indicates that, as with interpolation,
the maximum speedup for spread is limited by hardware, rather than being a
limitation of the algorithm.

The trendlines for these tests indicate that increasing computing resources
by a factor of two decreases runtime by a factor of about 1.87 for these
algorithms. Again, this is merely an approximation as the sort and reduction
steps of the spread algorithm are provided by {\thrust}, and therefore are
not limited to the listed number of threads. The similarity between the result
of the tests with RBCs and with randomly placed points indicates that the
distribution of points does not have a marked impact on the efficacy of the
parallelization for a fixed problem. We now see if the same holds for weak
scaling tests.


%\begin{table}
%    \begin{center}
%        \begingroup
%        \setlength{\tabcolsep}{9pt}
%        \renewcommand{\arraystretch}{1.5}
%        \begin{tabular}{ccccc}
%            $p$   & cells & \titletable{interpolate}{20000} & \titletable{spread}{10000} \\ \hline
%            64    & 4     & $0.04150 \pm 0.00013$           & $0.40148 \pm 0.00146$      \\
%            128   & 4     & $0.02251 \pm 0.00003$           & $0.20628 \pm 0.00060$      \\
%            256   & 4     & $0.01172 \pm 0.00002$           & $0.11208 \pm 0.00031$      \\
%            512   & 4     & $0.00603 \pm 0.00002$           & $0.06313 \pm 0.00014$      \\
%            1024  & 4     & $0.00349 \pm 0.00002$           & $0.03931 \pm 0.00011$      \\
%            2048  & 4     & $0.00181 \pm 0.00001$           & $0.02785 \pm 0.00011$      \\
%            4096  & 4     & $0.00103 \pm 0.00001$           & $0.02168 \pm 0.00009$
%        \end{tabular}
%        \endgroup
%    \end{center}
%    \caption{%
%        Results of strong scaling tests for IB spread and interpolate with
%        4 RBCs.
%        Reported values are in seconds.
%    }
%\end{table}
