\subsubsection{Weak scaling}

To see how the algorithms scale given more computing resources, we increase the
number of cells in the domain and the number of threads proportionally. We
construct each cell with $n_d=832$ data sites and $n_s=8832$ sample sites, as
before. We place between 1 and 8 cells in the domain, while threads increase
from 64 to 512. Using a timestep of $k = 0.1\us$, we simulate the motion of
these cells for $1\ms$.

In Section \ref{sec:unst-weak}, we observe that, as a side-effect of increasing
point density per grid cell, runtime for spreading decreases as the number of
points and threads increases. Here, the cells are initially far enough part as
to not have any overlapping support points.  As a result, while individual grid
cells may contain several IB points, average point density is still low, so we
do not expect to see the same reduction in runtime as observed previously.

\begin{table}[t]
    \begin{center}
        \begingroup
        \setlength{\tabcolsep}{9pt}
        \renewcommand{\arraystretch}{1.5}
        \begin{tabular}{ccccc}
%                                                                                           \toprule
%            $p$ & cells & \titletable{interpolate}{20000}  & \titletable{spread}{10000} \\ \midrule
%            64  & 1     & $0.01079 \pm 0.00007 $           & $0.11881 \pm 0.00107 $     \\
%            128 & 2     & $0.01165 \pm 0.00005 $           & $0.11219 \pm 0.00099 $     \\
%            256 & 4     & $0.01171 \pm 0.00005 $           & $0.11214 \pm 0.00070 $     \\
%            512 & 8     & $0.01199 \pm 0.00005 $           & $0.11354 \pm 0.00092 $     \\ \bottomrule
                                                                                           \toprule
            $p$ & cells & \titletable{interpolate}{20000} & \titletable{spread}{10000} \\ \midrule
            64  & 1     & $0.01079$                       & $0.11881$                  \\
            128 & 2     & $0.01165$                       & $0.11219$                  \\
            256 & 4     & $0.01171$                       & $0.11214$                  \\
            512 & 8     & $0.01199$                       & $0.11354$                  \\ \bottomrule
        \end{tabular}
        \endgroup
    \end{center}
    \caption{%
Weak scaling results for interpolate and spreading for increasing numbers of
RBCs (cells column) and threads. Each RBC has $n_d = 864$ and $n_s = 8832$.
%95\% confidence intervals for average time per call are reported in seconds.
Average time per call is reported in seconds.
$N$ is the number of samples taken.
    }
    \label{tab:str-weak}
\end{table}

Table \ref{tab:str-weak} shows the runtimes for increasing number of RBCs and
threads. We observe the near-perfect scaling we saw with random IB points. For
both interpolate and spread, we see that the runtimes are nearly constant: the
slowest and fastest times differ by less than $2\ms$ and $7\ms$, for
interpolation and spread, respectively. After an initial drop in runtime
between one RBC with 64 threads and two RBCs and 128 threads, runtimes even off
and begin to increase, in contrast with the results in Section
\ref{sec:unst-weak}.
