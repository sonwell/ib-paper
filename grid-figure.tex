\begin{figure}[t]
    \begin{center}
    \begin{tikzpicture}[scale=1.25]
        \draw[help lines] (-0.1, -0.1) grid (5.1, 5.1);
        \draw[color=gray, fill=gray, opacity=0.2] (0.6, 0.6) rectangle (4.6, 4.6);
        \draw[pattern={thick horizontal lines}, pattern color=gray, draw=none] (2, 2.5) rectangle (3, 3.5);
        \draw[thick, gray, fill=none] (2, 2.5) rectangle (3, 3.5);
        \draw[pattern=thick vertical lines, pattern color=gray, draw=none] (2.5, 2) rectangle (3.5, 3);
        \draw[thick, gray, fill=none] (2.5, 2) rectangle (3.5, 3);
        \node[circle, color=black, fill=black, inner sep=2pt] at (2.6, 2.6) {};

        \foreach \x in {0,...,5}
            \foreach \y in {0,...,4}
                \node[regular polygon, regular polygon sides=3, color=black, fill=none, inner sep=1.414pt, shape border rotate=-90, thick, draw] at (\x, \y+0.5) {};
        \foreach \x in {0,...,4}
            \foreach \y in {0,...,5}
                \node[regular polygon, regular polygon sides=3, color=black, fill=none, inner sep=1.414pt, thick, draw] at (\x+0.5, \y) {};
        \node[regular polygon, regular polygon sides=3, color=black, fill=black, inner sep=1.414pt, shape border rotate=-90] at (2, 2.5) {};
        \node[regular polygon, regular polygon sides=3, color=black, fill=black, inner sep=1.414pt] at (2.5, 2) {};
        \node[black] at (0.25, 4.75) {\sffamily(a)};
    \end{tikzpicture}
    
    \vspace{1em}

    \begin{tikzpicture}[scale=1.25]
        \draw[help lines] (-0.1, -0.1) grid (5.1, 5.1);
        \draw[color=gray, fill=gray, opacity=0.2] (1.1, 1.1) rectangle (4.1, 4.1);
        \draw[pattern=thick horizontal lines, pattern color=gray, draw=none] (2.5, 2) rectangle (3.5, 3);
        \draw[thick, gray, fill=none] (2.5, 2) rectangle (3.5, 3);
        \draw[pattern=thick vertical lines, pattern color=gray, draw=none] (2, 2.5) rectangle (3, 3.5);
        \draw[thick, gray, fill=none] (2, 2.5) rectangle (3, 3.5);
        \node[circle, color=black, fill=black, inner sep=2pt] at (2.6, 2.6) {};

        \foreach \x in {0,...,5}
            \foreach \y in {0,...,4}
                \node[regular polygon, regular polygon sides=3, color=black, fill=none, inner sep=1.414pt, shape border rotate=-90, thick, draw] at (\x, \y+0.5) {};
        \foreach \x in {0,...,4}
            \foreach \y in {0,...,5}
                \node[regular polygon, regular polygon sides=3, color=black, fill=none, inner sep=1.414pt, thick, draw] at (\x+0.5, \y) {};
        \node[regular polygon, regular polygon sides=3, color=black, fill=black, inner sep=1.414pt, shape border rotate=-90] at (3, 2.5) {};
        \node[regular polygon, regular polygon sides=3, color=black, fill=black, inner sep=1.414pt] at (2.5, 3) {};
        \node[black] at (0.25, 4.75) {\sffamily(b)};
    \end{tikzpicture}
    \end{center}

    \caption{%
A region of a 2-dimensional domain containing point $\X$, indicated by a black
circle. Light gray lines indicate physical units in increments of $h$, and $\X$
has the same coordinates in each subfigure. Light gray boxes indicate the
support of $\Dirac_h(\cdot-\X)$. Subfigure (a) has $s=4$ support points in each
dimension, and (b) has $s=3$. The grid for horizontal vector components has
staggering $\stag=(0,\,0.5)$ and is marked by right-pointing triangles, whereas
upright triangles mark the grid for vertical vector components and has
staggering $\stag=(0.5,\,0)$. Those grid points within the gray boxes are also
support points of $\X$. The horizontally- and vertically-striped gray boxes
denote the grid cell containing $\X$ on the horizontal or vertical component
grid, respectively. The filled triangles mark the point $\x=h(\idx{\X}+\stag)$,
corresponding to shift $\shift=(0,\,0)$ on the appropriate grid. The positions
of the grid cells in (a) are typical of all even $s$, and the positions of the
grid cells in (b) are typical of all odd $s$.
    }
    \label{fig:grid}
\end{figure}
