\begin{figure}[tb]
    \begin{center}
    \begin{tikzpicture}[scale=1]
        \draw[help lines] (-0.1, -0.1) grid (5.1, 5.1);
        \draw[color=gray, fill=gray, opacity=0.2] (0.6, 0.6) rectangle (4.6, 4.6);
        \draw[color=gray, fill=gray, opacity=0.4] (2, 2.5) rectangle (3, 3.5);
        \node[circle, color=black, fill=black, inner sep=2pt] at (2.6, 2.6) {};

        \foreach \x in {0,...,5}
            \foreach \y in {0,...,4}
            \draw[black, thick] (\x-0.1, \y+0.5) -- (\x+0.1, \y+0.5);
        \node[diamond, color=black, fill=black, inner sep=2pt] at (2, 2.5) {};
        \node[black] at (0.25, 4.75) {(a)};
    \end{tikzpicture}
    \hspace{1em}
    \begin{tikzpicture}[scale=1]
        \draw[help lines] (-0.1, -0.1) grid (5.1, 5.1);
        \draw[color=gray, fill=gray, opacity=0.2] (0.6, 0.6) rectangle (4.6, 4.6);
        \draw[color=gray, fill=gray, opacity=0.4] (2.5, 2) rectangle (3.5, 3);
        \node[circle, color=black, fill=black, inner sep=2pt] at (2.6, 2.6) {};

        \foreach \x in {0,...,4}
            \foreach \y in {0,...,5}
            \draw[black, thick] (\x+0.5, \y-0.1) -- (\x+0.5, \y+0.1);
        \node[diamond, color=black, fill=black, inner sep=2pt] at (2.5, 2) {};
        \node[black] at (0.25, 4.75) {(b)};
    \end{tikzpicture}

    \vspace{1em}

    \begin{tikzpicture}[scale=1]
        \draw[help lines] (-0.1, -0.1) grid (5.1, 5.1);
        \draw[color=gray, fill=gray, opacity=0.2] (1.1, 1.1) rectangle (4.1, 4.1);
        \draw[color=gray, fill=gray, opacity=0.4] (2.5, 2) rectangle (3.5, 3);
        \node[circle, color=black, fill=black, inner sep=2pt] at (2.6, 2.6) {};

        \foreach \x in {0,...,5}
            \foreach \y in {0,...,4}
            \draw[black, thick] (\x-0.1, \y+0.5) -- (\x+0.1, \y+0.5);
        \node[diamond, color=black, fill=black, inner sep=2pt] at (3, 2.5) {};
        \node[black] at (0.25, 4.75) {(c)};
    \end{tikzpicture}
    \hspace{1em}
    \begin{tikzpicture}[scale=1]
        \draw[help lines] (-0.1, -0.1) grid (5.1, 5.1);
        \draw[color=gray, fill=gray, opacity=0.2] (1.1, 1.1) rectangle (4.1, 4.1);
        \draw[color=gray, fill=gray, opacity=0.4] (2, 2.5) rectangle (3, 3.5);
        \node[circle, color=black, fill=black, inner sep=2pt] at (2.6, 2.6) {};

        \foreach \x in {0,...,4}
            \foreach \y in {0,...,5}
            \draw[black, thick] (\x+0.5, \y-0.1) -- (\x+0.5, \y+0.1);
        \node[diamond, color=black, fill=black, inner sep=2pt] at (2.5, 3) {};
        \node[black] at (0.25, 4.75) {(d)};
    \end{tikzpicture}
    \end{center}

    \caption{%
A region of a 2-dimensional domain $\domain$ containing point $\X$, indicated
by a black circle. Light grey lines indicate physical units, and $\X$ has the
same coordinates in each subfigure. Light grey boxes indicate the support of
$\Dirac_h(\cdot-\X)$. Subfigures (a) and (b) have $s=4$ support points in each
dimension, and (c) and (d) have $s=3$. The horizontal dashes in (a) and (c)
intersect the unit lines at grid points on a grid with staggering
$\stag=(0,\,0.5)$ for the horizontal component of a vector-valued quantity, and
the intersection of vertical dashes and unit lines in (b) and (d) mark grid
points on a grid with staggering $\stag=(0.5,\,0)$ for the vertical component.
Those grid points within the grey boxes are also support points of $\X$. Dark
grey boxes denote the grid cell containing $\X$, and black diamonds mark the
grid point $\x=h(\idx{\X}+\stag)$, which corresponds to shift $\shift=(0,\,0)$.
The position of the grid cell in (a) and (b) is typical of all even $s$, and
the position of the grid cell in (c) and (d) is typical of all odd $s$.
    }
    \label{fig:grid}
\end{figure}
