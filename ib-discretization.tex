\subsection{The structure of $\mathcal{S}$ and $\mathcal{S}^\dagger$}

Let $\Omega_h$ be a regular grid on $\Omega$ with grid spacing $h$. Define
$\vec{g} \in [0,\,1)^d$ to be the fixed vector such that a grid point $\vec{x}$
can be decomposed as $\vec{x}=h(\vec{i}+\vec{g})$, where $\vec{i}$ has integer
components. Moreover, let $\Gamma_h$ be a discretization of the interface
$\Gamma$. For illustration purposes, we can think of $\Gamma_h$ as a collection
of arbitrary points in $\Omega$. We refer to points in $\Omega_h$ as (Eulerian)
grid ponts, and points in $\Gamma_h$ as Lagrangian points.

[Equations @eq:scalar-interp;@eq:scalar-spread] are discretized on $\Omega_h$
and $\Gamma_h$, respectively, to yield
\begin{align}
    \label{eq:disc-interp}
    E(\vec{X}_j) &= \sum_i \delta_h(\vec{x}_i-\vec{X}_j)e(\vec{x}_i) h^d \quad \text{and} \\
    \label{eq:disc-spread}
    \ell(\vec{x}_i) &= \sum_j \delta_h(\vec{x}_i-\vec{X}_j)L(\vec{X}_j) \d A.
\end{align}
Each of the above equations look like a matrix-vector multiplication, so we 
will define $\mathcal{S}=(\delta_h(\vec{x}_i-\vec{X}_j))$, where the subscript
$i$ indicates the row and subscript $j$ indicates the column. Collecting the
values of $e(\vec{x})$ at each Eulerian grid point and of $L(\vec{X})$ at
each Lagrangian point, we rewrite [equations @eq:disc-interp;@eq:disc-spread]
as
\begin{align}
    \label{eq:matrix-interp}
    \vec{E} &= \mathcal{S}^\dagger\vec{e} \quad\text{and} \\
    \label{eq:matrix-spread}
    \vec{\ell} &= \mathcal{S}\vec{L},
\end{align}
respectively. For this reason, we call $\mathcal{S}$ the \emph{spread matrix},
and its transpose, $\mathcal{S}^\dagger$, the \emph{interpolation matrix}.

As we have said, $\delta_h$ is the tensor product of scaled, one-dimensional
kernels, $h^{-1}\phi(h^{-1}x)$. Let $\mathrm{supp}\mskip\thinmuskip\phi$
denote the support of $\phi$ and define
\begin{equation}
    s[\phi] = |\mathrm{supp}\mskip\thinmuskip\phi\cap\mathbb{Z}|-1
\end{equation}
to be the size of the support in meshwidths. For brevity, we write $s=s[\phi]$.
For any $\vec{X}\in\Omega$, there are at most $s^d$ grid points
$\vec{x}\in\Omega_h$ for which $\delta_h(\vec{x}-\vec{X})$ is nonzero.

Let $\vec{y}\in\Omega$ be an arbitrary point and consider the set of grid
points for which $\delta_h(\vec{x}-\vec{y})$ and $\vec{x}\in\Omega_h$. Denote
this set of grid points $\Sigma(\vec{y})$, called the \emph{support points}
of $\vec{y}$. The pre-image $\Sigma^{-1}(\Sigma(\vec{y}))$ is a subset of
$\Omega$ containing at most one grid point. For $\vec{y}$ sufficiently far away
from any boundary, $\Sigma^{-1}(\Sigma(\vec{y}))$ is an $h \times h$ ($d=2$) or
$h \times h \times h$ ($d=3$) subset of $\Omega$. For points near a boundary,
the region may be smaller. Collectively, these regions cover $\Omega$, so we
consider them to be the \emph{de facto} grid cells. For those grid cells that
do not contain a grid point, we extend $\Omega_h$, in a regular way, with ghost
points. Call this extension $\bar{\Omega}_h$. Now, any $h \times h$ or
$h \times h \times h$ extended grid cell that entirely contains a grid cell
also contains exactly one grid point in $\bar{\Omega}_h$, including grid cells
near a boundary. We can identify an $\vec{y}\in\Omega$ with a grid point
$\vec{x}\in\bar{\Omega}_h$ if they are in the same extended grid cell, and
since $\vec{x}=h(\vec{i}=\vec{g})$, we identify a grid cell by the integers
$\vec{i}$. Finally, we define $\lfloor\cdot\rceil:\Omega\to\mathbb{Z}$ such
that $\lfloor\vec{y}\rceil = \vec{i}$.

We now turn our attention to the evaluation of $\delta_h(\vec{x}-\vec{X})$. We
assume $\vec{x}\in\Omega_h$ and write
\begin{equation}
    \label{eq:delta-defs}
    \begin{aligned}
        \delta_h(\vec{x}-\vec{X})
        &= \delta_h(\vec{x}-h(\lfloor\vec{X}\rceil+\vec{g}) + h(\lfloor\vec{X}\rceil+\vec{g}) - \vec{X}) \\
        &= \delta_h(h\vec{\sigma} - \Delta\vec{X}) \\
        &= \prod_{i=1}^d h^{-1}\phi((\vec{\sigma} - h^{-1}\Delta\vec{X})\cdot\vec{c}_i).
    \end{aligned}
\end{equation}
where $\Delta\vec{X}$ is the displacement of $\vec{X}$ from its associated grid
point, $\vec{\sigma} = \lfloor\vec{x}\rceil-\lfloor\vec{X}\rceil$ since
$\vec{x} = h(\lfloor\vec{x}\rceil+\vec{g})$, and $\vec{c}_i$ is the $i^\text{th}$
canonical basis vector. We refer to $\vec{\sigma}$ as a \emph{shift}. Shifts
that result in a possibly nonzero value of $\delta_h$ are known \emph{a priori}
based on $\phi$, and usually range from $-\lfloor s/2\rfloor$ to
$\lfloor(s-1)/2\rfloor$ in each component. We can therefore choose an order for
the shifts $\{\vec{\sigma}_i\}_{i=1}^{s^d}$. We denote the $i^\text{th}$ shift
$\vec{\sigma}_i$.

We need one more ingredient to construct $\mathcal{S}$. Let $\vec{x}_k$ be the
be the $k^\text{th}$ grid point, such that, e.g., $e_k = e(\vec{x}_k)$ is the $k^\text{th}$
entry of $\vec{e}$. The grid point $\vec{x}_k$ can be decomposed into
$h(\vec{i}+\vec{g})$ for some $\vec{i}$ with integer components. Define the
grid indexing function $\#:\mathbb{Z}^d\to\mathbb{Z}\cup\{\epsilon\}$ such that
$\#(\lfloor\vec{x}_k\rceil) = \#(\vec{i}) = k$ for all grid points and
$\#(\vec{i}')=\epsilon$ if $h(\vec{i}'+\vec{g}) \not\in \Omega$.

We are now ready to construct $\mathcal{S}$. Consider a Lagrangian point
$\vec{X}_j$ that is in the same grid cell as grid point
$\vec{x}_k=h(\lfloor\vec{X}_j\rceil+\vec{g})$. The $j^\text{th}$ column of
$\mathcal{S}$ is zero except for up to $s^d$ values where for
$i=1,\,\ldots,\,s^d$, if $\#(\lfloor\vec{X}_j\rceil+\vec{\sigma}_i)\neq\epsilon$,
\begin{equation}
    \label{eq:s-columnwise}
    \mathcal{S}_{i,\#(\lfloor\vec{X}_j\rceil + \vec{\sigma}_i)} = \delta_h(h\vec{\sigma}_i-\vec{X}_j+\vec{x}_k).
\end{equation}
