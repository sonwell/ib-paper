\subsection{Structure of IB interaction operators}

Let $\domain_h$ be a regular grid on $\domain$ with grid spacing $h$. Define
$\vec{g} \in [0,\,1)^d$ to be the fixed vector such that a grid point $\x$
can be decomposed as $\x=h(\vec{i}+\vec{g})$, where $\vec{i}$ has integer
components. Moreover, let $\interface_h$ be a discretization of the interface
$\interface$. For illustration purposes, we can think of $\interface_h$ as a
collection of arbitrary points in $\domain$. We refer to points in $\domain_h$
as (Eulerian) grid ponts, and points in $\interface_h$ as (Lagrangian) IB
points.

Equations \eqref{eq:scalar-interp} and \eqref{eq:scalar-spread} are discretized
on $\domain_h$ and $\interface_h$, respectively, to yield
\begin{align}
    \label{eq:disc-interp}
    E(\X_j) &= \sum_i \delta_h(\x_i-\X_j)e(\x_i) h^d \quad \text{and} \\
    \label{eq:disc-spread}
    \ell(\x_i) &= \sum_j \delta_h(\x_i-\X_j)L(\X_j) \d\gamma_j.
\end{align}
Each of the above equations look like a matrix-vector multiplication, so we 
define the \emph{spread matrix} $\mathcal{S}=(\delta_h(\x_i-\X_j))$, where the
subscript $i$ indicates the row and subscript $j$ indicates the column. We call
its transpose, $\mathcal{S}^\dagger$, the \emph{interpolation matrix}.
Collecting values $e(\x_i)h^d$ at each grid point and $L(\X_j)\d\gamma_j$ at
each IB point, we rewrite Equations \eqref{eq:disc-interp} and
\eqref{eq:disc-spread} in matrix form as
\begin{align}
    \label{eq:matrix-interp}
    \vec{E} &= \mathcal{S}^\dagger\vec{e} \quad\text{and} \\
    \label{eq:matrix-spread}
    \vec{\ell} &= \mathcal{S}\vec{L},
\end{align}
respectively.

The kernel $\delta_h$ is the tensor product of scaled, one-dimensional
kernels, $h^{-1}\phi(h^{-1}x)$. Let $\mathrm{supp}\mskip\thinmuskip\phi$
denote the support of $\phi$ and define
\begin{equation}
    s[\phi] = |\mathrm{supp}\mskip\thinmuskip\phi\cap\mathbb{Z}|-1
\end{equation}
to be the size of the support in units of grid spaces. For brevity, we write
$s=s[\phi]$.  For any $\X\in\domain$, there are at most $s^d$ grid points
$\x\in\domain_h$ for which $\delta_h(\x-\X)$ is nonzero. We call these points
the \emph{support points} of $\X$. If $\X$ is sufficiently far from any
boundary, the subset of $\domain$ which has the same support points as $\X$ is
an $h^d = h\times\cdot\times h$ region, and contains a single grid point. For
$\X$ near a boundary, the region may be smaller, and may not contain a grid
point, depending on the type of boundary. We imagine extending the domain such
that each of these regions has side length $h$ in each dimension. We extend the
grid with ghost points so that each region contains a single grid point. We
call this extension $\bar{\domain}_h$. Collectively, these regions cover
$\domain$, so we consider them to be the \emph{de facto} grid cells, and
henceforth refer to them as such. We can now identify any $\X\in\domain$ with a
grid point $\x\in\bar{\domain}_h$ if they are in the same grid cell, and since
$\x=h(\vec{i}+\vec{g})$, we identify a grid cell by the integers $\vec{i}$. We
adopt the notation $\idx{\X}=\vec{i}$ for the function that maps points in
$\domain$ to the vector of integers identifying the grid cell containing $\X$.

We now turn our attention to the evaluation of $\delta_h(\x-\X)$. We assume
$\x\in\domain_h$ and write
\begin{equation}
    \label{eq:delta-defs}
    \begin{aligned}
        \delta_h(\x-\X)
        &= \delta_h(\x-h(\idx{\X}+\vec{g}) + h(\idx{\X}+\vec{g}) - \X) \\
        &\equiv \delta_h(h\shift - \Delta\X) \\
        &= \prod_{i=1}^d h^{-1}\phi((\shift - h^{-1}\Delta\X)_i).
    \end{aligned}
\end{equation}
where $\Delta\X = \X-h(\idx{\X}+\vec{g})$ is the displacement of $\X$ from its
associated grid point, $\shift = \idx{\x}-\idx{\X}\in\mathbb{Z}^d$ since
$\x \equiv h(\idx{\x}+\vec{g})$, and subscript $i$ denotes the $i^\text{th}$
component. We refer to $\shift$ as a \emph{shift}. Shifts that result in
a possibly nonzero value of $\delta_h$ are known \emph{a priori} based on the
kernel $\phi$, and usually range from $-\floor{s/2}$ to $\floor{(s-1)/2}$ in
each component. We can therefore assign an order to the shifts, $\shift_1$,
$\shift_2$, \dots, $\shift_{s^d}$. We need only compute $\Delta\X$ once to be
able to compute several values of $\delta_h$.

We need one more ingredient to construct $\mathcal{S}$. Let $\x_k$ be the be
the $k^\text{th}$ grid point. That is, for Eulerian function $e(\x)$, we
construct the vector $\vec{e}$ with $k^\text{th}$ entry $e_k = e(\x_k)$.
Suppose that $\idx{\x_k}=\vec{i}$. Define the \emph{grid indexing function}
$\#:\mathbb{Z}^d\to\mathbb{Z}\cup\{\epsilon\}$ such that $\#(\idx{\x_k}) =
\#(\vec{i}) = k$ for $\x\in\domain_h$ and $\#(\vec{i}') = \epsilon$ if
$h(\vec{i}'+\vec{g}) \not\in \domain$. The value $\epsilon$ is used as an error
value and distinguishes an Eulerian point which does not have a corresponding
row in $\mathcal{S}$, and therefore does not contribute to, e.g., $\vec{\ell}$.
We are now ready to construct $\mathcal{S}$. Let $\X_j$ be an IB point. The
$j^\text{th}$ column of $\mathcal{S}$ is zero except for up to $s^d$ values
where, for $i=1,\,\ldots,\,s^d$, if $\#(\idx{\X_j}+\shift_i)\neq\epsilon$,
\begin{equation}
    \label{eq:s-columnwise}
    \mathcal{S}_{\#\left(\idx{\X_j} + \shift_i\right),j} = \delta_h\left(h\shift_i-\X_j+\idx{\X_j}\right).
\end{equation}

