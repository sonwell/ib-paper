\subsection{Preliminaries}

For a regular grid on $\domain$ with grid spacing $h$, $\domain_h$, any grid point
$\x\in\domain_h$ can be decomposed into a vector $\vec{i}$ with integer components and a
fixed grid staggering, $\stag\in[0,\,1)^d$, such that $\x=h(\vec{i}+\stag)$. Let
$\interface_h$ be a discretization of the interface $\interface$. For the purposes of
interpolation and spreading, we can think of $\interface_h$ as a collection of arbitrary
points in $\domain$.

\begin{figure}[tb]
    \begin{center}
    \begin{tikzpicture}[scale=1]
        \draw[help lines] (-0.1, -0.1) grid (5.1, 5.1);
        \draw[color=gray, fill=gray, opacity=0.2] (0.6, 0.6) rectangle (4.6, 4.6);
        \draw[color=gray, fill=gray, opacity=0.4] (2, 2.5) rectangle (3, 3.5);
        \node[circle, color=black, fill=black, inner sep=2pt] at (2.6, 2.6) {};

        \foreach \x in {0,...,5}
            \foreach \y in {0,...,4}
            \draw[black, thick] (\x-0.1, \y+0.5) -- (\x+0.1, \y+0.5);
        \node[diamond, color=black, fill=black, inner sep=2pt] at (2, 2.5) {};
        \node[black] at (0.25, 4.75) {(a)};
    \end{tikzpicture}
    \hspace{1em}
    \begin{tikzpicture}[scale=1]
        \draw[help lines] (-0.1, -0.1) grid (5.1, 5.1);
        \draw[color=gray, fill=gray, opacity=0.2] (0.6, 0.6) rectangle (4.6, 4.6);
        \draw[color=gray, fill=gray, opacity=0.4] (2.5, 2) rectangle (3.5, 3);
        \node[circle, color=black, fill=black, inner sep=2pt] at (2.6, 2.6) {};

        \foreach \x in {0,...,4}
            \foreach \y in {0,...,5}
            \draw[black, thick] (\x+0.5, \y-0.1) -- (\x+0.5, \y+0.1);
        \node[diamond, color=black, fill=black, inner sep=2pt] at (2.5, 2) {};
        \node[black] at (0.25, 4.75) {(b)};
    \end{tikzpicture}

    \vspace{1em}

    \begin{tikzpicture}[scale=1]
        \draw[help lines] (-0.1, -0.1) grid (5.1, 5.1);
        \draw[color=gray, fill=gray, opacity=0.2] (1.1, 1.1) rectangle (4.1, 4.1);
        \draw[color=gray, fill=gray, opacity=0.4] (2.5, 2) rectangle (3.5, 3);
        \node[circle, color=black, fill=black, inner sep=2pt] at (2.6, 2.6) {};

        \foreach \x in {0,...,5}
            \foreach \y in {0,...,4}
            \draw[black, thick] (\x-0.1, \y+0.5) -- (\x+0.1, \y+0.5);
        \node[diamond, color=black, fill=black, inner sep=2pt] at (3, 2.5) {};
        \node[black] at (0.25, 4.75) {(c)};
    \end{tikzpicture}
    \hspace{1em}
    \begin{tikzpicture}[scale=1]
        \draw[help lines] (-0.1, -0.1) grid (5.1, 5.1);
        \draw[color=gray, fill=gray, opacity=0.2] (1.1, 1.1) rectangle (4.1, 4.1);
        \draw[color=gray, fill=gray, opacity=0.4] (2, 2.5) rectangle (3, 3.5);
        \node[circle, color=black, fill=black, inner sep=2pt] at (2.6, 2.6) {};

        \foreach \x in {0,...,4}
            \foreach \y in {0,...,5}
            \draw[black, thick] (\x+0.5, \y-0.1) -- (\x+0.5, \y+0.1);
        \node[diamond, color=black, fill=black, inner sep=2pt] at (2.5, 3) {};
        \node[black] at (0.25, 4.75) {(d)};
    \end{tikzpicture}
    \end{center}

    \caption{%
A region of a 2-dimensional domain $\domain$ containing point $\X$, indicated
by a black circle. Light grey lines indicate physical units, and $\X$ has the
same coordinates in each subfigure. Light grey boxes indicate the support of
$\Dirac_h(\cdot-\X)$. Subfigures (a) and (b) have $s=4$ support points in each
dimension, and (c) and (d) have $s=3$. The horizontal dashes in (a) and (c)
intersect the unit lines at grid points on a grid with staggering
$\stag=(0,\,0.5)$ for the horizontal component of a vector-valued quantity, and
the intersection of vertical dashes and unit lines in (b) and (d) mark grid
points on a grid with staggering $\stag=(0.5,\,0)$ for the vertical component.
Those grid points within the grey boxes are also support points of $\X$. Dark
grey boxes denote the grid cell containing $\X$, and black diamonds mark the
grid point $\x=h(\idx{\X}+\stag)$, which corresponds to shift $\shift=(0,\,0)$.
The position of the grid cell in (a) and (b) is typical of all even $s$, and
the position of the grid cell in (c) and (d) is typical of all odd $s$.
    }
    \label{fig:grid}
\end{figure}


The kernel $\Dirac_h$ is the tensor product of scaled, one-dimensional kernels,
$h^{-1}\kernel(h^{-1}x)$. Let $\supp\kernel$ denote the support of $\kernel$ and define
\begin{equation}
    s[\kernel] = \max_{r\in[0,\,1)}|\supp\kernel(\cdot-r)\cap\mathbb{Z}|-1
\end{equation}
to be the size of the support in terms of unit intervals. For brevity, we write
$s=s[\kernel]$. For any $\X\in\domain$, there are at most $s^d$ grid points
$\x\in\domain_h$ for which $\Dirac_h(\x-\X)$ is nonzero. We call these points the
\term{support points} of $\X$. If $\X$ is sufficiently far from any boundary, the subset
of $\domain$ each of whose points has the same support points as $\X$ is an
$h^d=h\times\cdots\times h$ region, and contains a single grid point. For $\X$ near a
non-periodic boundary, the region may be smaller, and may not contain a grid point. We
extend the grid so that each of these regions has side length $h$ in each dimension and
contains a single grid point.  We call this extension $\bar{\domain}_h$. Collectively,
these regions cover $\domain$, so we consider them to be the \emph{de facto} grid cells,
and henceforth refer to them as such. Figure \ref{fig:grid} illustrates these grid cells,
and the effect different choices of $\kernel$ has on their location. We can now identify
any $\X\in\domain$ with a grid point $\x\in\bar{\domain}_h$ if they are in the same grid
cell, and since $\x=h(\vec{i}+\stag)$ is the only grid point in the grid cell, we
identify the cell by the integers $\vec{i}$. We adopt the notation $\idx{\X}=\vec{i}$ for
the function that maps points in $\domain$ to the vector of integers identifying the grid
cell containing $\X$.

We now turn our attention to the evaluation of $\Dirac_h(\x-\X)$. We assume
$\x\in\domain_h$ and write
\begin{equation}
    \label{eq:delta-defs}
    \begin{aligned}
        \Dirac_h(\x-\X)
        &= \Dirac_h(\x-h(\idx{\X}+\stag) + h(\idx{\X}+\stag) - \X) \\
        &\equiv \Dirac_h(h\shift - \Delta\x) \\
        &= \prod_{i=1}^d h^{-1}\kernel((\shift - h^{-1}\Delta\x)_i),
    \end{aligned}
\end{equation}
where $\Delta\x = \X-h(\idx{\X}+\stag)$ is the displacement of $\X$ from its associated
grid point, $\shift=\idx{\x}-\idx{\X}$, and subscript $i$ denotes the $i$ component. We
refer to $\shift$ as a \term{shift}. Shifts that result in a possibly nonzero value of
$\Dirac_h$ are known \emph{a priori} based on the kernel $\kernel$, and usually range
from $-\floor{s/2}$ to $\floor{(s-1)/2}$ in each component. We can therefore assign an
order to the shifts, $\shift_1$, $\shift_2$, \dots, $\shift_{s^d}$. We need only compute
$\Delta\x$ once to be able to compute all nonzero values of $\Dirac_h(\x-\X)$.

We need one more ingredient to construct $\spread$. We let $\x_k$ be the be the $k\th$
grid point, and for Eulerian function $e(\x)$, we construct the vector $\vec{e}$ with
$k\th$ entry $e_k = e(\x_k)$. Define the \term{grid indexing function}
\begin{equation}
    \label{eq:grid-index-fn}
    \#(\vec{i}) = \begin{cases}
        k, & \idx{\x_k}=\vec{i},\ \x_k\in\domain_h \\
        \error, & \text{otherwise}.
    \end{cases}
\end{equation}
The value $\error$ indicates an Eulerian point outside of $\domain$, does not have a
corresponding row in $\spread$, and therefore does not contribute to spreading or
interpolation. We are now ready to construct $\spread$. Let $\X_j$ be an IB
point. The $j\th$ column of $\mathcal{S}$ is zero except for up to $s^d$ values where,
for $i=1,\,\ldots,\,s^d$, if $\#(\idx{\X_j}+\shift_i)=k\neq\error$,
\begin{equation}
    \label{eq:s-columnwise}
    \spread_{kj} = \Dirac_h\left(h\left(\shift_i+\idx{\X_j}+\stag\right)-\X_j\right).
\end{equation}
From a practicality standpoint, it is unnecessary to explicitly construct $\spread$. We
illustrate this with the serial spreading algorithm.

% vim: cc=90 tw=89
