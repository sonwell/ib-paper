\subsection{Surface geometry via radial basis function interpolation}

Fix a set of distinct points $\{\vec{\theta}_i\}_{i=1}^{n_d}$ in parameter
space. We refer to these $n_d$ points as \textit{data sites} and are used to
construct an interpolant which will acts as the surface of a cell. To
accomplish this, we track the coordinates in $\mathbb{R}^3$ of the point on the
surface corresponding to each of the data sites.

We can choose a set of distinct points in parameter space, called the
\textit{sample sites}, which are possibly different from the data sites, at
which to evaluate the interpolant and its derivatives, and ultimately to
compute the Lagrangian force density. These points will be used to spread force
density from the surface to the Eulerian grid, so we choose them to satisfy the
IB method heuristic that the mesh size $h_s$ satisfies $h_s \lesssim h/2$,
where $h$ is the Eulerian grid size.

The surface interpolant is constructed from a linear combination of radial
basis functions and another polynomial basis. While generally, these basis
functions may not be true polynomials, we will refer to them as such for 
simplicity of illustration. Details ... [@Shankar:...].

Let $\vec{X}_k$ be the coordinates of the point on the cell surface
corresponding to data site $\vec{\theta}_k$. The interpolant should recover
these coordinates exactly, for each data site. In other words, we want
\begin{equation}
    \vec{X}_k = \sum_{i=1}^{n_d} b_i\varphi(d(\vec{\theta}_k,\,\vec{\theta}_i)) + \sum_{j=1}^{n_p} c_jp_j(\vec{\theta}_k),
\end{equation}
where $\varphi$ is called the \textit{basic function}, $d$ is a suitable metric
for the object under study, and $n_p$ is the number of elements in the set of
polynomial basis functions $\{p_j\}_{j=1}^{n_p}$. For certain choices of
$\varphi$, $n+p\ge 1$ may be required [@Wendland:...;@Fasshauer:...]. For an
individual component of $\vec{X}$ at each data site, this yields $n_d$
conditions in $n_d+n_p$ variables. To uniquely determine the interpolant, we
require further that the polynomial basis recover polynomial data:
\begin{equation}
    0 = \sum_{i=1}^{n_d} b_ip_j(\vec{\theta}_i).
\end{equation}
Collectively, these conditions can be written
\begin{equation}
    \label{eq:rbf-interpolant}
    \left[\begin{array}{cc}\Phi & P \\ P^T & 0\end{array}\right]
    \left[\begin{array}{c}\vec{b}\\\vec{c}\end{array}\right] =
    \left[\begin{array}{c}\vec{X}\\\vec{0}\end{array}\right].
\end{equation}
Assuming that the system is uniquely determined, we have constructed our
interpolant.

Let $\mathcal{L}$ be a linear operator for which $\Psi=(\mathcal{L}\varphi(d(\vec{\theta},\,\vec{\theta}_i)))$,
for $i=1,\,\ldots,\,n_d$, and $Q=(\mathcal{L}p_j(\vec{\theta}))$, for $j=1,\,\ldots,\,n_p$,
can be computed analytically. For the purposes of computing Lagrangian force
density, of particular interest are the identity (or "evaluation") operator,
and the various first and second partial derivative operators. With
a sufficiently smooth choice of $\varphi$, we can approximate $\mathcal{L}\vec{X}(\vec{\theta})$
at an arbitrary point in parameter space by applying the operator $\mathcal{L}$
to the interpolant and evaluating the result. That is,
\begin{equation}
    \mathcal{L}\vec{X}(\vec{\theta})
    \approx \sum_{i=1}^{n_d} b_i\mathcal{L}\varphi(d(\vec{\theta},\,\vec{\theta}_i))
    + \sum_{j=1}^{n_p} c_j\mathcal{L}p_j(\vec{\theta})
\end{equation}
by linearity. But the coefficients $\vec{b}$ and $\vec{c}$ are determined by
the interpolant [@eq:rbf-interpolant], so
\begin{align}
    \mathcal{L}\vec{X}(\vec{\theta})
    &= \left[\begin{array}{cc}\Psi&Q\end{array}\right]
    \left[\begin{array}{cc}\Phi&P\\P^T&0\end{array}\right]
    \left[\begin{array}{c}\vec{X}\\\vec{0}\end{array}\right] \\
    &= \left[\begin{array}{cc}L&\ast\end{array}\right]
    \left[\begin{array}{c}\vec{X}\\\vec{0}\end{array}\right],
\end{align}
where $L\vec{X}$ is now the discrete analogue to $\mathcal{L}\vec{X}(\vec{\theta})$,
i.e., the operator that applies $\mathcal{L}$ and evaluates at $\vec{\theta}$.
The $n_p$ columns represented by $\ast$ are usually ignored, as they are
always multiplied by a vector of zeros and therefore do not contribute.

We also require integration weights, which act as surface patch areas in
computing surface forces from a surface force density. In many cases, the
integral
\begin{equation}
    I_\varphi(\vec{\theta}') = \int_\Gamma \varphi(d(\vec{\theta},\,\vec{\theta}'))\mskip\thinmuskip\mathrm{d}\hat{\vec{X}}(\vec{\theta})
\end{equation}
may not be known for topological archetype $\Gamma$. But
following the work of [@Narcowitz:...], for certain surfaces,
$I_\varphi(\vec{\theta}')$ is independent of $\vec{\theta}'$ and is therefore
constant. Assuming $I_\varphi = \text{constant}$ and
\begin{equation}
    I_1 = \int_\Gamma \mathrm{d}\hat{\vec{X}}(\vec{\theta})
\end{equation}
is known, we can compute integration weights $\hat{\vec{a}}$ for $\Gamma$ via
\begin{equation}
    \left[\begin{array}{cc}\Phi &\vec{1} \\ \vec{1}^T & 0\end{array}\right]
    \left[\begin{array}{c}\hat{\vec{a}}\\-I_\varphi\end{array}\right] =
    \left[\begin{array}{c}\vec{0}\\I_1\end{array}\right].
\end{equation}
A side effect of solving this system is that we obtain an approximation to
the constant $I_\varphi$.

To compute integration weights for a shape $\vec{X}(\vec{\theta})$, which is
topologically equivalent to $\hat{\Gamma}$, we compute the
Jacobian for $\hat{\Gamma}$ and the cell,
\begin{equation}
    \hat{V}=\left|\frac{\partial\hat{\vec{X}}}{\partial\theta^i}\right| \quad \text{and} \quad
    V=\left|\frac{\partial\vec{X}}{\partial\theta^i}\right|.
\end{equation}
Since $\hat{\Gamma}$ is known, we can compute $\hat{V}$ analytically. The
expression $a_k/\hat{V}(\vec{\theta}_k)$ gives the integration weight in
parameter space and can be computed without any information about the cell,
except that the cell and $\Gamma$ are topologically equivalent. From these
weights and an interpolant for the cell, the expression
$\mathrm{d}A_k=a_k V(\vec{\theta}_k)/\hat{V}(\vec{\theta}_k)$ gives the
integration weight on the surface of the cell.

